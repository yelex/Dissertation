\chapter{Описание механизма сбора и обработки данных по ценам}\label{ch:ch2}
В России, как и во всем мире, в последние годы наблюдается тенденция к наращиванию объемов интернет-торговли. Данные Ассоциации компаний интернет-торговли показывают, что объем интернет-торговли в России за последние 9 лет вырос практически в 10 раз  --- с 713 млрд рублей в 2014 году до 6,3 трлн рублей в 2023 году. Развитие сегмента ускорилось в период пандемии, когда население оказалось в условиях коронавирусных ограничений, что привело к росту онлайн-покупок. Рост интернет-торговли привел также к росту данных по ценам на отдельные товары и услуги, доступные к покупке через интернет, что существенно расширило возможности исследователей для сбора данных и анализа поведения цен на микроуровне. В настоящей главе будет описана методика сбора данных, характеристики собранных данных, а также трудности, с которыми исследователь может столкнуться в процессе сбора данных и возможные пути решения этих трудностей с учетом накопленного опыта.
\section{Методика сбора данных по ценам онлайн-ритейлеров}\label{sec:ch2/sec1}

Механизм сбора данных работает следующим образом: исследователи совместно с техническими специалистами разрабатывают программные коды на Python, которые ежедневно запускаются на удаленном сервере. Эти программы автоматически собирают информацию о текущей цене товара, скидочной цене (если товар продается по сниженной цене), названии товара, ссылке на товар, а также в некоторых случаях о дополнительных характеристиках, таких как вес, цвет, размер упаковки и другие.

Одним из важных преимуществ онлайн-данных является их относительно низкая стоимость сбора. Написанная однажды программа и расходы на обслуживание сервера значительно дешевле, чем содержание штата сотрудников, которые должны регулярно посещать точки продаж и вносить цены в базу данных. Кроме того, вероятность совершения ошибки из-за человеческого фактора минимизируется, так как человек не участвует в процессе сбора данных. Программа автоматически собирает данные, однако требуется контроль за корректностью собранных данных. Это не вызывает существенных затрат в плане когнитивных и физических усилий. Важным преимуществом собранных данных является их высокая частотность, что позволяет отслеживать ценовые тенденции практически в режиме реального времени, особенно на площадках онлайн-ритейла.

Процесс сбора онлайн-данных о ценах устроен следующим образом. Ежедневно или с любой другой частотой программа анализирует и извлекает данные из разметки сайта, содержащей информацию о товаре и его стоимости. Если разметка меняется, возникает ошибка, требующая обновления кода. Следует отметить, что по опыту автора такие ошибки возникают редко. Затем информация обрабатывается и проверяется на ошибки, чтобы соответствовать формату базы данных, и заносится в нее.

Главным недостатком собранных онлайн-данных является малое покрытие категорий и точек продаж. Но последние исследования показывают, что это недостаток постепенно устраняется. Например, доля онлайн-продаж в общем обороте розничной торговли в России значительно увеличилась с 2\% в 2019 году до 9,2\% в 2021 году, согласно данным ассоциации интернет-торговли.

Выбор сайтов, с которых собирается информация, является важным аспектом. Если исследователь стремится определить тенденции в офлайн-сегменте по ценам в онлайн-ритейле, рекомендуется отдать предпочтение мультиканальным ритейлерам, которые предлагают продукцию как онлайн, так и в офлайн точках продаж. В таких случаях цены на товары обычно совпадают в 72\% случаев и изменения цен происходят с одинаковой частотой и амплитудой.

Для достижения репрезентативности важно учитывать размеры ритейлеров и выбирать те, которые занимают значительную долю в расходах потребителей. Также следует сосредоточиться на категориях товаров, входящих в традиционную корзину потребления домохозяйств.

Что касается выбора сайтов для сбора данных о товарах, рекомендуется предпочитать онлайн-ритейлеров, так как другие источники, такие как агрегаторы и третьи сайты, не всегда предоставляют актуальную информацию о товарах и их характеристиках. Нельзя гарантировать, что сделки проводятся по ценам, указанным на таких сайтах-агрегаторах.

\section{Методологические проблемы сбора данных и возможные способы их решения}\label{sec:ch2/sec2}


\subsection{Проблема выбора ритейлеров}\label{subsec:ch2/sec2/sub1}

Главная проблема, с которой исследователь сталкивается при сборе данных из интернета, заключается в определении круга сайтов онлайн-ритейлеров, с которых необходимо собирать соответствующую информацию. Выбор этих сайтов сильно зависит от цели формирования базы данных. 

Если цель состоит в том, чтобы создать индекс цен, который позволит сделать наиболее точные выводы о поведении цен в обычных традиционных магазинах и приблизить официальную инфляцию, то, по нашему мнению, имеет смысл придерживаться методологии, представленной в \cite{cavallo2016billion}. Эта методология предполагает сбор данных о ценах с сайтов ритейлеров, которые продажают одни и те же товары как онлайн, так и офлайн, но преимущественно последним способом. Источник \cite{cavallo2016online} показывает, что динамика цен на товары у таких ритейлеров в онлайн- и офлайн-сегментах во многом схожа, что позволяет с некоторой уверенностью делать выводы о поведении цен в традиционных точках продаж на основе этих данных. На практике мы столкнулись с тем, что доля мультиканальных ритейлеров среди всех онлайн-ритейлеров в российских условиях не очень высока, но мы старались включать их в базу данных насколько это было возможно.

\subsection{Замещение наблюдений вследствие исчезновения товаров из продажи }\label{subsec:ch2/sec2/sub2}

Для расчета индекса потребительских цен во времени необходимо иметь непрерывные данные для каждого товара или услуги. Однако иногда возникают ситуации, когда цены на наблюдаемые товары недоступны по разным причинам. В таких случаях требуется заменить отсутствующие данные, чтобы обеспечить сопоставимость индекса цен во времени.

На самом низком уровне агрегирования (на уровне товаров и услуг в конкретном городе) Росстат собирает данные о ценах на 5-10 товаров и услуг с определенными характеристиками. Например, для товара "молоко питьевое цельное пастеризованное 2,5-3,2\% жирности"~могут собираться данные о различных марках молока с жирностью 3,2\%. Если цена на конкретный товар отсутствует, Росстат рекомендует собирать информацию о ценах на более широкий круг товаров с похожими потребительскими свойствами. Это позволяет использовать дополнительные данные при подборе замены для отсутствующих товаров.

Кроме того, Росстат предлагает множество методов замены цен на отсутствующие товары в зависимости от природы товара и причин его отсутствия. Все эти методы позволяют обеспечивать непрерывность рядов с индексом цен во времени, но требуют сложных процедур и принятия решений для каждой конкретной ситуации исчезновения товара.

Однако сбор данных с сайтов онлайн-ритейлеров позволяет значительно снизить проблему замещения отсутствующих данных, так как это не требует больших затрат. Наш опыт подтверждает, что чем больше товаров с определенными свойствами собирается, тем ближе совокупный индекс цен по этой группе к официальному индексу.

\subsection{Отнесение товара к категории}\label{subsec:ch2/sec2/sub3}
