\chapter{Обзор эмпирических исследований, посвященных изучению динамики цен онлайн-ритейлеров}\label{ch:ch1}

\section{Эволюция исследований по использованию онлайн-данных в изучении отдельных аспектов ценовой динамики}\label{sec:ch1/sec1}

Одной из первых работ, посвященных изучению поведения цен интернет-ритейлеров, стало исследование \cite{RePEc:ecb:ecbwps:2006645}. Основная цель исследования состояла в том, чтобы оценить различие между степенью жесткости цен в интернет-магазинах и традиционных (т.е. офлайн-) магазинах, а также получить оценки жестоксти для различных стран, типов товаров, и определить, от каких факторов зависит жесткость цен.

Авторы использовали уникальную базу данных по 5 миллионам ценовых котировок, собранных с сайтов-агрегаторов 4-х европейских стран (Франция, Италия, Германия и Великобритания) и США за период с декабря 2004 года по декабрь 2005 года. Данные скачивались на ежедневной основе. На момент сбора данных интернет-торговля была не слишком широко развита и покрывала в основном непродовольственные товары. Набор данных, собранных авторами, покрывал потребительскую электронику (DVD-плееры, телевизоры, домашние мини-системы), развлекательную электронику (портативные mp3-плееры, цифровые видеокамеры), компьютерную технику (ноутбуки, сканеры), кухонную технику (микроволновые печи, кофемашины), мелкую бытовую технику (пылесосы), крупную бытовую технику (холодильники, стиральные машины) и услуги (фотопроявка). Использование фотопроявки в качестве примера конкретной услуги было продиктовано тем, что услуги в Европе на тот момент были в целом слабо представлены в интернете, и кроме того, фотопроявка являлась достаточно четко определенной категорией услуг, что позволяет сравнивать свойства жесткости цен по ней между различными странами. Практически половина от собранных авторами данных (48\%) приходится на данные из США.

Далее в работе описываются стилизованные факты относительно жесткости цен в Европе и США. Авторы отмечают, что несмотря на низкие издержки изменения, цены не менялись на ежедневной основе. Последнее наблюдалось для всех стран и категорий товаров. В среднем на всей выборке (т.е. по всем категориям и рассматриваемым странам) ежедневная частота изменений цен составила 2,6\% (т.е. 2,6\% ценовых котировок менялись ежедневно). На уровне стран наибольшая частота изменений цен наблюдалась в Италии (около 3,8\%), а наименьшеая – в Великобритании (2,1\%). Средняя частота изменений цен во Франции составила 3,1\%, в Германии – 2,7\%, в США – 2,5\%. Для большей интерпретируемости результатов авторы использовали понятие дюрации, или среднего периода неизменности цен ($D^{av}_{k}$), который рассчитывается, отталкиваясь от средней частоты ценовых изменений:

\begin{equation}
	\label{eq:equation1}
	D_k^{av}=\frac{-1}{ln⁡(1-F_k)},
\end{equation}
где \( k \) "--- идентификатор страны или категории товара, \( F_k \) "--- средняя частота изменения.

Авторы получили разброс оценок средней дюрации от 25 дней во Франции до 68 дней – в США. В среднем для 4-х европейских стран авторы получили оценку в 31 день – то есть цены по используемым в работе данным остаются неизменными в среднем месяц. Авторы отмечают, что эти результаты противоречат оценкам по США на традиционных данных [2], где был получен период неизменности в 6-7 месяцев. Вероятно, более низкая оценка жесткости, полученная авторами, является следствием в целом более высокой частоты изменений в интернете и использованием данных дневной, а не месячной частоты.

На уровне отдельных товаров также наблюдается гетерогенность – так, минимальная средняя частота изменений варьировалась от 1,3\% в день для кофемашин до 4,3\% - для ноутбуков. После кофемашин, наименьшей средней частотой характеризовались микроволновые печи (1,5\% в день), домашние мини-системы (1,5\%), пылесосы (1,6\%) и холодильники (1,6\%). Авторы предполагают, что низкая частота изменений цен на эти категории товаров продиктована относительно более длительным сроком службы и меньшей скоростью морального устаревания по сравнению с остальными товарами, чьи цены в среднем меняются чаще (например, ноутбуками). Стоит также отметить, что среди всех сочетаний продуктовых категорий-исследуемых стран наименьшей частотой изменения характеризовались микроволновые печи в США (0,4\% в день), а наибольшей – телевизоры в Великобритании (6,3\%).

Авторы отдельно отмечают категорию фотопроявки. За весь рассматриваемый период из 13000 ценовых котировок, собранных для европейских стран, было зафиксировано лишь 14 изменений цен. Этот результат может свидетельствовать о том, что торговля в онлайн-ритейле приводит к повышению гибкости цен в сфере товаров, но не услуг. Сфера услуг в целом отличается более жесткими ценами по сравнению с обычными потребительскими товарами, что было показано, к примеру, на данных США \cite{bils2004some}.

В работе также отмечается, что гетерогенность в жесткости цен между странами менее выражена по сравнению с гетерогенностью между отдельными категориями товаров и услуг. Так, средняя частота изменений цен по категориям, как было отмечено выше, варьируется от 1,3\% (для кофемашин) до 4,3\% (для ноутбуков) в день, в то время как между странами средняя частота варьируется от 3,4\% (для Франции) до 5,6\% (для Великобритании). Нужно сказать, что разница в частотах и на отдельные категории товаров (за исключением микроволновых печей, холодильников и пылесосов) и услуг не сильно варьируется между отдельными странами. 

Авторы отмечают, что снижения цен в интернете случаются чаще, чем было найдено по данным европейских офлайн-цен \cite{dhyne2005price}. В работе \cite{dhyne2005price} отмечалось, что лишь 4 из 10 ценовых изменений являются отрицательными. Авторы работы \cite{RePEc:ecb:ecbwps:2006645} на данных онлайн-цен обнаружили большую долю снижений: от 40\% для микроволновых печей в Великобритании до 87\% на телевизоры (также в Великобритании). В среднем по всем странам и категориям, доля снижений цен составила 62\%. Вместе с тем авторы отмечают, что столь высокая доля снижений может объясняться особенностями товаров в используемой выборке: поскольку основная доля товаров – это электроника с высоким темпом морального устаревания – то частое снижение цен перед введением новой модели является широко распространенным явлением, увеличивающим долю снижений цен на общем фоне.

Авторами также была обнаружена высокая степень гетерогенности между типами магазинов. Так, компании, которые занимаются доставкой товаров по почте, а также ТВ-магазины имеют в среднем наименьшую частоту изменений цен. Этот результат согласуется с концепцией издержек меню, поскольку эти два типа магазинов несут наибольшие издержки при изменении цен по сравнению с остальными типами магазинов. Тот же аргумент объясняет, почему магазины, продающие товары только онлайн, имеют наибольшую частоту изменений цен.

В работе было также обнаружено, что средний размер изменений цен является относительно высоким и составляет 5,4\%. Вместе с тем, как отмечают авторы, это число меньше, чем оценки, полученные на традиционных офлайн-данных. Интервал, в котором происходило большинство изменений цен в онлайн-ритейле, составляет от 0 до -1\%, что разнится с данными традиционных офлайн-ритейлеров (у которых лишь малая доля изменений находится в промежутке от -2,5\% до 2,5\% \cite{Hoffmann2006}). Последнее также соотносится с предсказаниями концепции издержек меню, поскольку издержки изменения цен у онлайн-ритейлеров ниже, эти изменения для них дешевле, и потому они могут корректировать их чаще.

Наконец, авторы построили панельную логит-модель, в которой изменения цен попытались объяснить как факторами, зависящими от времени (time-dependent variables), так и факторами, зависящими от рыночной обстановки (state-dependent variables). 

Результаты модели показали, что частота изменений цен увеличивается с увеличением числа продавцов, предлагающих этот продукт, а также с увеличением доли изменений цен на этот товар, произошедших в предыдущий день. Привлекательные цены (такие как €9,99), а также относительно высокие цены снижают вероятность их изменения при прочих равных. В целом, эти результаты оказываются устойчивыми для всех подвыборок товаров, продающихся во всех рассматриваемых странах.

Ключевым в области изучения поведения цен онлайн-ритейлеров стал проект The Billion Prices Project. Проект значительно отличается от предыдущих попыток сбора данных по ценам онлайн-ритейлеров, поскольку является более широкомасштабным, методологически совершенным и устойчивым во времени. Подробное описание проекта приводится в работе \cite{cavallo2016billion}.

Авторы проекта отмечают, что его появление было мотивировано манипулированием данных по инфляции в Аргентине с 2007 по 2015 гг. К 2007 году стало понятно, что официально публикуемый уровень инфляции в Аргентине значительно отличается от того, что на самом деле происходит с ценами: это показывали как расчеты местных экономистов, так и опросы домохозяйств. Авторы будущего проекта стали собирать данные по ценам на ежедневной основе и показали, что на фоне официально заявляемой ежегодной инфляцией за 2007-2011 гг. в 8\% данные по ценам онлайн-ритейлеров демонстировали среднюю ежегодную инфляцию в 20\%. Авторы на данных онлайн-ритейлеров также показали, что манипуляция с официально публикуемой инфляцией завершилась в декабре 2015 года, с избранием нового правительства в Аргентине. Таким образом, случай с обнаружением значительных статистических расхождений между онлайн- и офлайн-инфляцией показал, что данные по ценам онлайн-ритейлеров обладают существенным потенциалом для использования в измерении инфляции.

Последнее привело к созданию The Billion Prices Project в Массачусетском технологическом институте – проекта по сбору цен онлайн-ритейлеров для нескольких стран, включая США. К 2010 году в проекте уже собиралось около 5 миллионов цен порядка 300 ритейлеров из 50 стран мира. Несмотря на то, что собирать цены в интернете значительно дешевле, чем традиционным офлайн-способом, проект столкнулся с проблемами финансирования, что привело к созданию коммерческого ответвления PriceStats – компании, которая предоставляет данные по высокочастотным индексам для центральных банков и клиентов финансового сектора.

Авторы отмечают важность методологии сбора данных. Авторы тщательно отбирают ритейлеров, используемых как источники данных, используют технологии «веб-скраппинга» для сбора данных, затем производят очистку и приводят данные в соответствие с целями исследований или измерения инфляции. 

Стоит остановиться на первом этапе методологии. Несмотря на огромный массив данных, критически важным для целей измерения и прогнозирования инфляции является тщательный отбор как категорий, так и ритейлеров. Ключевой целью авторов проекта является репрезентативный сбор транзакций. При отборе ритейлеров авторы стараются игнорировать ритейлеров, которые продают свои товары исключительно онлайн, и сосредотачиваются в основном на мультиканальных ритейлерах – то есть ритейлерах, которые продают товары как через интернет, так и традиционным офлайн-способом (речь идет о таких магазинах, как Walmart). Как поясняют авторы проекта, главная причина такого внимания к мультиканальным ритейлерам состоит в том, что они в подавляющем большинстве стран мира являются вовлеченными в большинство транзакций, что важно с точки зрения репрезентативности ценовых индексов. Авторы также отмечают, что при сборе данных по ценам внутри таких ритейлеров они сосредотачиваются, как правило, на тех категориях товаров, которые являются частью официальной корзины национального индекса потребительских цен, и стараются избежать товаров, которые чрезмерно представлены в онлайн-ритейле (речь о таких товарах, как CD-, DVD-диски, косметика и книги).

После сбора данных авторы приступают к их очистке, стандартизации для соответствия общей схеме базы данных, классификации отдельных товаров по категориям индекса потребительских цен и расчету простых характеристик. Каждый из ритейлеров является уникальной «стратой» с уникальными характеристиками и ценовым поведением. Перед тем, как включить ритейлера в процесс сбора данных для расчета индекса цен, авторы мониторят поведение ритейлера в течение года для идентификации любых специфичных характеристик в собираемых данных чтобы понять, насколько в целом будет полезным включение данного ритейлера для расчета ценовых индексов.

Авторы отмечают, что объем данных и покрытие различных категорий отличается между странами. Для приблизительно 25 стран собираемые авторами данные покрывают как минимум 70\% весов локальных индексов потребительских цен.

Следующим после отбора источников данных шагом является процесс непосредственного сбора данных ценовой информации. Авторы используют технологию «веб-скраппинга», которая с течением времени значительно улучшилась. Если раньше «веб-скраппинг» требовал от исследователей написания программ на таких языках как Python или PHP, то сегодня существует много «point-and-click» систем, позволяющих без специальных навыков программирования обучить программу на сбор данных с определенных частей страницы. Такое программное обеспечение позволяет создать робота, который будет способен извлекать нужную информацию из веб-сайта с однородной структурой и помещать эту информацию в базу данных. Вызовом для сбора данных является обнаружение ошибок, возникающих с течением времени (например, из-за изменения разметки сайта). Авторы собирают следующие данные: название товара, его описание, бренд, размер, информацию о категории и цену (если доступно, то еще информацию о том, отсутствует ли товар и является ли цена распродажной).

Отдельно авторы сосредоточились на преимуществах и недостатках собираемых данных. Авторы производили сравнение с традиционными данными, лежащими в основе расчета национального индекса потребительских цен, а также данными, собираемыми независимыми агентствами, такими как AC Nielsen. 

Одним из главных преимуществ онлайн-данных по ценам является низкая стоимость наблюдений. Как отмечают авторы, издержки сбора хотя и не тривиальны, однако являются гораздо более низкими, чем оплата труда сотрудников, которые будут физически посещать магазины, или стоимость наблюдений у таких провайдеров как AC Nielsen.

Еще одним достоинством онлайн-данных является высокая частота наблюдений - чаще всего, дневная, однако существует возможность собирать данные с любой возможной частотой. Также это преимущество позволяет избежать усреднения по времен, что является частым явлением в сборе традиционных офлайн-данных по ценам.

Третьим важным преимуществом является наличие детализированной информации для всех товаров в выборке ритейлеров. Как правило, это преиущство еще и дополняется большим объемом ценовых котировок, собираемых внутри категорий, чем в случае традиционных данных (в этом случае собирается, как правило, 5-10 котировок на категорию). Такое преимущество позволяет избежать проблему оценки изменения качества при исчезновении одного товара и появлении другого.

В-четвертых, данные онлайн-ритейлеров не имеют цензурированных рядов цен. Цены на товары собираются до тех пор, пока товар не исчезнет из магазина. Традиционные методы напротив, часто начинают наблюдать новые товары только тогда, когда исчезнет предыдущие, и таким образом информация о более ранней ценовой истории товара не наблюдается.

Пятое преимущество состоит в том, что данные онлайн-ритейлеров собираются удаленно. Авторы указывают, что такое преимущество становится особенно ценным как в случае с Аргентиной, в которой правительство пыталось всячески помешать независимому сбору данных для расчета инфляции. Это преимущество также позволяет сделать сбор данных централизованным и более однородным с точки зрения характеристик собираемых данных.

Шестое преимущество следует из пятого и состоит в том, что наборы собранных данных могут быть прямо сопоставлены между отдельными странами, поскольку методы сбора данных для разных стран применяются одни и те же. Это особенно полезно в исследовательских ценлях сопоставления между отдельными странами, товарами и временными периодами.

Наконец, данные онлайн-ритейлеров являются доступными в режиме реального времени, с отсутствием какой-либо задержки в доступе и обработке информации. Это преимущество особенно важно для лиц, принимающих оперативные решения в области денежно-кредитной политики и для всех специалистов, нуждающихся в максимально оперативной информации о ценах.

Одним из основных недостатков собираемых данных является гораздо меньшее покрытие ритейлеров и продуктовых категорий, чем в случае собираемой национальными органами статистики. В частности, цены большинства услуг все еще остаются недоступными в интернете, и кроме того число и разнообразие ритейлеров остается ограниченным по сравнению с данными официальной статистики.

Еще одним недостатком онлайн-данных по ценам является отсутствие информации о количестве проданных товаров. До сих пор данные онлайн-ритейлеров сочетаются с весами официальной статистики для использования в расчетах, требующих этих весов. Авторы отмечают, что данные независимых агентств, таких как AC Nielsen, напротив, имеют детализированную информацию о количестве проданных товаров, и потому могут быть потенциально использованы как источник высокочастотных данных о весах в некоторых категориях товаров, например в продовольствии.

Авторы отдельно останавливаются на вопросе о том, являются ли цены онлайн- и офлайн- различными. Этот вопрос является важным, поскольку распространение выводов из онлайн- на офлайн-торговлю требует предпосылки, что цены в целом ведут себя одинаково (а для некоторых выводов требуется также, чтобы и уровни цен совпадали). В работах \cite{brynjolfsson2000frictionless, ellison2009search, gorodnichenko2018price} авторы показали, что в онлайн-ритейле цены, по-видимому, меняются чаще и на меньшую величину, чем цены, лежащие в основе расчета индекса потребительских цен. Однако, ритейлеры, которые использовались в вышеупомянутых исследованиях, как правило продавали свою продукцию только через интернет, что отличается от используемых в работе данных по мультиканальным ритейлерам.

Чтобы понять, насколько отличаются цены и их динамика в онлайн- и офлайн-ритейле для мультиканальных ритейлеров, в работе \cite{cavallo2016online} были описаны данные по ценам на 24000 товаров, которые собирались одновременно онлайн- и офлайн в 56 крупнейших ритейлерах 10 стран. Сопоставление было реализовано путем создания специального программного обеспечения для регистрации цен в физических магазинах, привлечения волонтеров и техник «веб-скраппинга». Прямое сопоставление показало как высокую степень схожести в уровнях цен, так и в частоте и размере изменений цен. Результаты показали, что 70\% уровней цен на один и тот же товар схожи в онлайн- и офлайн-ритейле. Изменения цен хотя и не были точно синхронизированны, однако размер и частота изменений в целом были схожи между собой. Стоит заметить, что отсутствие синхронизации может в целом приводить к тому, что изменения онлайн-цен, вероятно, могут предсказывать изменения цен в офлайн-рителе.

Часть исследования \cite{cavallo2016billion} была посвящена изучению построения индексов инфляции на данных онлайн-ритейлеров. Авторы на примере нескольких латиноамериканских стран показывают, что данные по ценам онлайн-ритейлеров могут быть качественным альтернативным источником для построения индексов, которые будут демонстрировать близкую динамику к официальным индексам цен. Различие возникает в основном в уровне инфляции, но не в динамики индекса с течением времени. Авторы также показывают, что онлайн-индекс значительно быстрее, чем официальный, реагирует на агрегированные шоки.

Кроме того, синхронность онлайн- и офлайн-индексов инфляции для разных стран оказывается различным. Авторы показывают, что для США эти индексы демонстрируют очень близкую между собой динамику в течение 7 лет наблюдений. Столь высокая степень сходства объясняется относительно высокой (по сравнению с другими странами) долей интернет-торговли в общем объеме розничной торговли США, а также высокой долей мультиканальных ритейлеров в общем торговом обороте. В целом индексы, построенные на данных онлайн-ритейлеров, достаточно близко реплицируют поведение официальной инфляции, что справедливо как для крупных, так и для малых стран, для развитых и для развивающихся рынков.

Авторы отдельно сосредотачивают внимание на возможности онлайн-индекса цен предсказывать будущее направление и амплитуду изменений в офлайн-инфляции после того или иного шока. Так, в 2008 году, после банкротства Lehman Brother’s в США онлайн-индекс продемонстрировал драматическое падение практически сразу, в то время как официальному индексу потребовалось 2 месяца, чтобы показать тенденцию инфляции к замедлению.

Важной проблемой для сбора данных на длинную перспективу является т.н. «перекрытие качества» (overlapping quality). Проблема состоит в том, что при исчезновении того или иного товара из продажи его требуется заменить на новый товар, однако для сохранения сопоставимости ряда во времени нужно нивелировать различие в качестве между старым и новым товарами. Поскольку возможности статистических органов по сбору данных не такие широкие, как в случае сбора данных онлайн-ритейлеров, то на каждую категорию товаров им приходится собирать сильно ограниченное количество товаров или услуг. В большинстве национальных статистических ведомств используются различные техники для аппроксимации динамики в момент перехода от одного товара к другому, которые помогают сгладить различия в качестве заменяемых товаров. Часто для этих целей используются гедонистические регрессии, однако зачастую они имеют ряд ограничений, связанных с оценкой качества товаров. Данные онлайн-ритейлеров в этих ситуациях имеют преимущество, поскольку из-за большого числа собираемых марок/моделей/уникальных товаров в рамках каждой из категорий исчезновение одного из товаров не сильно влияет на общую динамику цен. 

Как было отмечено раннее, онлайн-индекс полезен для предсказания официальной инфляции. Для документирования этого факта авторы построили авторегрессионную модель с распределенным лагом, где в качестве зависимой переменной был использован официальным ИПЦ США и онлайн-индекс цен в качестве независимой переменной, и рассчитали импульсные отклики, чтобы посмотреть, как шоки онлайн-индека влияют на офлайн-индекс с течением времени. Были использованы ежемесячные разности как в случае зависимой, так и независимой переменной, а также 6 лагов запаздывания каждой из переменных. Результаты показали, что для США традиционный офлайн-индекс требует нескольких месяцев для учета информации из изменения онлайн-индекса. На уровне отдельных секторов наиболее быстрый эффект наблюдается в области цен на топливо и наиболее медленный – в сфере продуктов и электроники. Авторы подчеркивают, что способность онлайн-индексов предсказывать поведение офлайн-индекса может объясняться задержкой в публикации данных традиционных индексов, различиями в выборках магазинов, а также более быстрой адаптацией цен в онлайн-ритейле для некоторых отдельных секторов. 

Еще одним приложением данных по ценам онлайн-ритейлеров является уточнение выводов и оценок по жесткости цен. Последнее является фундаментальным элементом многих макроэкономических моделей. В последние десятилетия появился большой объем эмпирической литературы, посвященной оценке тех или иных аспектов поведения цен на микроуровне и изучению оснований этого поведения (см., к примеру, \cite{bils2004some,klenow2008state,gagnon2009}). Эти исследования стали возможны благодаря бесперецедентному доступу исследователей к базам данных по ценам, лежащим в основе расчета национальных индексов потребительских цен.

Вместе с тем, работа \cite{cavallo2018scraped} на данных онлайн-ритейлеров показала, что полученные выводы являются смещенными из-за характеристик данных по ценам, используемым для построения индексов потребительских цен, а также данных независимых агентств. В частности, автор показал, что два свойства этих данных ведут к смещению: усреднение цен во времени (характерно для данных независимых агентств) и специфическая для данных по ценам официальных статистических ведомств замена пропущенных цен. Автор подчеркивает, что онлайн-данные по ценам лишены подобных проблем, поскольку однажды настроенная программа по сбору цен собирает их на ежедневной основе (таким образом, отсутстве усреднение в конце недели) и без замены пропущенных цен какими-либо методами (как в случае данных национальных статистических ведомств).

Используемый авторами онлайн-датасет включает в себя более 60 миллионов ежедневных наблюдений по ценам в пяти странах: Аргентина, Бразилия, Чили, Колумбия и США. Данные были собраны с сайтов 8 разных компаний за период с 2007 по 2010 гг. Для США использовались данные 4 крупнейших ритейлеров в стране: супермаркет, гипермаркет/универмаг, аптека и ритейлер, который продает в основном электронику. В других странах использовались данные крупнейшего ритейлера в стране. Все эти ритейлеры являются лидерами в своих странах с рыночной долей около 28\% в Аргентине, 15\% в Бразилии, 27\% в Чили и 30\% в Колумбии. Авторы отмечают, что в данных встречается большое количество пропущенных наблюдений (в основном вследствие ошибок в «веб-скраппинге» или временном отсутствии товара на складе). В зависимости от страны, доля пропущенных значений варьируется от 22\% до 37\% от всех наблюдений. Часть изменений цен оказалась слишком большой по амплитуде, что является следствием ошибок в процессе сбора данных. По этой причине из данных были удалены изменения свыше 200\% и меньше -70\%.

Чтобы показать эффект усреднения, который применяется в данных агентства Nielsen, автор сопоставляет динамику цен для одного и того же ритейлера, его местоположения и временного периода онлайн- и офлайн-. Автор на данных онлайн-ритейлеров производит симуляцию этих же данных, используя для каждого товара еженедельное усреднение по ценам, что производит близкое соответствие данным Nielsen. Еженедельное усреднение приводит к тому, что одно изменение цен превращается в два последовательных изменения цен, что порождает более частые и меньшие по размеру изменения цен, что полностью меняет распределение ценовых изменений. Кроме того, это приводит к тому, что функция риска (важный динамический показатель поведения цен) искажается и становится полностью убывающей – с искусственным пиком изменений на первой неделе.

Автор показал, что средний период неизменности цен и размер изменений сокращается примерно на 50\%, и такой эффект наблюдается для всех стран. Этот результат объясняет полученные в литературе результаты обнаружения очень гибких цен. Автор обнаружил, что для США средний период неизменности цен составляет 1,53 месяца, против оценок на данных независимых агентств, варьирующихся от 0,6 до 1 месяца. Использование симуляции усреднения на онлайн-данных породило средний период изменения, равный 0,8 месяцам, что является средним значением между 0,6 и 1 месяцам. Чтобы сделать сравнение более точным, автором были приобретены данные независимого агентства для одного и того же ритейлера, локации и временного периода на онлайн-данных и данных этого агентства. Автор получил оценку в 0,8 месяцев, что соответствует усреднению по онлайн-данным для того же самого ритейлера, почтового индекса и временного периода.

Вместе с тем, среднему периоду изменений цен в литературе уделяется меньше внимания, чем распределению изменений цен. Автор показывает, что распределение изменений цен в онлайн-ритейле является в большей степени бимодальным, с очень малым количество изменений, близких к 0. Данные независимых агентств, напротив, демонстрируют одномодальное распределение изменений цен с очень высокой долей изменений, близких к 0. Автор отмечает, что такое распределение является превалирующим в литературе и послужило мотивацией к созданию моделей ценообразования, учитывающей малые изменения цен. Таким образом, еженедельное усреднение цен может объяснять различие в полученных выводах. Из-за усреднения цен, количество малых по модулю изменений становится значительно больше, что приводит к одномодальности распределения. Автор также подчеркивает, что малая доля околонулевых изменений и наличие двух мод соответствует предсказаниям гипотезе издержек меню, главной идеей которой является отсутствие малых изменений из-за их неоптимальности.

Наконец, усреднение во времени влияет на оценки функций риска. Функции риска показывают зависимость вероятности изменения цены от срока ее неизменности. Функция является важной с точки зрения определения того, какая из моделей ценообразования является более адекватной наблюдаемым данным. Модели с «издержками меню» предполагают, что функция риска является возрастающей, поскольку со временем все больше товаров оказываются за границами своего «бездействия» и выгоды от изменения цены вынуждают все большее число фирм принимать решение об изменении цен, таким образом вероятность изменения цен со временем растет. Модели, в которых ценообразование зависит от времени, напротив, предполагают либо отсутствие взаимосвязи между сроком изменения цен и вероятности изменения (как в модели Кальво), либо генерируют пики вероятности в определенные периоды «жизни» цены (как в модели Тейлора). Усреднение цен во времени приводит к тому, что функция риска становится строго убывающей, с высоким значением вероятности изменения цен в первую неделю. Напротив, функция риска, построенная на данных онлайн-ритейлеров имеет форму горба: сначала вероятность изменения цен растет со временем, затем постепенно падает. Увеличение вероятности изменения цен в начале функции также совпадает с выводами моделей издержек меню в том, что чем дольше цена остается неизменной, тем больше она отклоняется от оптимальной цены (в условиях устойчивых шоков, таких как накопленная инфляция), и тем выше становится вероятность изменения цен.

Авторы также остановились на эффектах, которые производит замещение пропущенных наблюдений по ценам на данных по расчету ИПЦ. Эти данные не подвержены усреднению во времени, как было описано выше, поскольку собираются один раз в месяц сотрудниками статистических ведомств. Однако, в случае отсутствия определенного товара в наличии, цена на него не регистрируется и заменяется определенным способом. Отсутствие наблюдений может быть связано с заменой товара на дргой товар (из-за его исчезновения из продажи) или его временной недоступностью (отсутствие на складе). 

Для замены пропущенных наблюдений многие статистические ведомства используются специальный метод – т.н. «относительное вменение по ячейкам» (cell-relative imputation). В рамках этого подхода цены на отсутствующий товар заменяются предыдущей ценой этого товара, умноженной на среднюю динамику внутри категории. Авторы указывают, что такой подход может механически приводить к увеличению частоты изменения цен, а также снижать размер этих изменений.

Чтобы проиллюстрировать эффект замещения пропущенных наблюдений средней динамикой по категории автор произвел симуляцию отсутствия наблюдений на данных онлайн-ритейлеров. Были отобраны цены на 15-ое число каждого месяца. Затем для каждого товара пропущенные цены были заменены предыдущей ценой, умноженной на геометрическое среднее изменений внутри той же категории товаров.

Результаты симуляции показали, что подход замещения пропущенных наблюдений драматически снижает дюрацию, или среднюю продолжительность неизменности цен. Ежемесячная дюрация сократилась с 4,7 месяцев до 3,35 месяцев, что близко к оценкам на данных, лежащих в основе расчета ИПЦ для США \cite{klenow2008state}. Таким образом, оценки частоты изменений цен на данных ИПЦ завышены из-за использования такого подхода замещения пропущенных наблюдений.

Эффект замещения пропущенных наблюдений также отражается и на размерах изменений цен. «Относительное вменение по ячейкам» приводит к тому, что средний размер изменений цен по модулю сокращается в США с 20,8\% до 16,2\%, а также увеличивается доля изменений цен ниже 1\% и 5\% по модулю, что также близко к оценкам \cite{klenow2008state}. В качестве дополнительной иллюстрации автор приводит график распределения изменений цен на ежемесячных онлайн-данных с симуляцией и без симуляции относительного вменения по ячейкам. Иллюстрация показывает, что, как и в случае с усреднением цен во времени, главным последствием замещения приводит к росту доли малых изменений цен. В случае использования такого замещения распределение цен становится одномодальным, тогда как без симуляции оно является бимодальным, что драматически искажает выводы о действительном поведении цен.

Также, как и в случае усреднения во времени, относительное вменение по ячейкам отражается и на функции риска порождением в ней ярко выраженного нисходящего наклона. Использование замещения пропущенных наблюдений приводит к возникновению двух последовательных изменений цен, что увеличивает вероятность изменения цены в первый месяц ее «жизни». 

Автор указывает, что выраженность эффектов смещения на настоящих данных ИПЦ может быть, однако, ниже, чем продемонстрировано им на симуляционных онлайн-данных. Одна из причин состоит в том, что доля пропущенных наблюдений в симуляции выше, чем заявленная в работе \cite{klenow2008state} (в последней эта доля равна 7\%). Во-вторых, не все пропущенные значения цен заменяются методикой «относительного вменения по ячейкам». Тем не менее, проведенная автором статьи симуляция указывает на то, что следует учитывать вышеуказанные особенности данных ИПЦ при оценке на их основе статистик жесткости цен.

\section{Эволюция исследований, посвященных использованию данных онлайн-ритейлеров для прогнозирования ценовых индексов}\label{sec:ch1/sec2}

Одним из направлений использования данных по ценам онлайн-ритейлеров стало прогнозирование офлайн-инфляции. Одной из первых работ, посвященных использованию онлайн-данных в таком ключе, стало исследование \cite{aparicio2020forecasting}. Авторы на данных онлайн-ритейлеров показали, что использование таких данных позволяет превзойти прогностические свойства различных традиционных офлайн-моделей, а также опросы профессиональных аналитиков.

Авторы использовали данные по ценам онлайн-ритейлеров компании PriceStats (коммерческое ответвление проекта The Billion Prices Project \cite{cavallo2016billion}) по следующим странам: Австралия, Канада, Франция, Германия, Греция, Ирландия, Италия, Нидерланды, Великобритания и США за период с июля 2008 г. по сентябрь 2016 г. Основная цель авторов состояла в предсказании значений месячного сезонно несглаженного индекса потребительских цен, рассчитанного для каждой из указанных стран. Авторы также указывают, что дневная частота данных онлайн-ритейлеров является преимуществом перед другими мерами инфляции, поскольку помогают обнаружить изменения в инфляционных трендах раньше конца месяца. Еще одно преимущество состоит в том, что индексы онлайн-ритейлеров публикуются сразу, тогда как как показатели ИПЦ обычно публикуются спустя 15 дней после сбора информации.

Модель, используемая авторами для прогнозирования ИПЦ, предлагается в следующем виде:

\begin{equation}
	\label{eq:equation2}
	E_{t-1}p_t = \hat{\alpha} + \sum_{i=1}^p \hat{\beta}_{t-i} p_{t-i} + \sum_{i=0}^p \hat{\theta}_{t-i} f_{t-i} + \sum_{i=0}^p \hat{\gamma}_{t-i} o_{t-i} + \sum_{i=0}^p \hat{\eta}_{t-i} of_{t-i},
\end{equation}
где \( p_t \) "--- индекс инфляции, \( f_t \) "--- индекс инфляции, построенный на данных по офлайн-ценам на бензин и дизельное топливо, \( o_t \) "--- онлайн-индекс инфляции, \( of_t \) "--- онлайн-индекс инфляции, построенный на данных по офлайн-ценам на бензин и дизельное топливо. Использование данных по дизельному топливу и бензину объясняется тем, что эти индексы по этим данным традиционно считаются хорошим предиктором для инфляции.

Оценка уравнения \cref{eq:equation2} сравнивается с 5 моделями-бенчмарками. Первая модель – это \cref{eq:equation2} без учета индексов, построенных на онлайн-данных. Сравнение с этой моделью позволяет оценить общее влияние добавления онлайн-индикаторов на прогнозирование инфляции. Второе сравнение проводится с прогнозами на 1 месяц вперед, опубликованным Bloomberg – прогноз, который является одним из самых известных прогнозов на рынке. Наконец, производится сравнение с тремя моделями, являющимися традиционными бенчмарками в литературе: AR(p), случайное блуждание (RW) и кривая Филиппса.

Авторы оценивают прогноз на один, два и три месяца вперед, и затем на основе этих прогнозов рассчитывают квартальный и годовой прогноз инфляции в США. Сравнение результатов модели \cref{eq:equation2} с результатами прогнозов вышеназванных бенчмарков показало, что качество прогнозов на основе модели с включением данных онлайн-ритейлеров значительно превосходит прогностические свойства моделей на традиционных офлайн-данных. Таким образом, в работе была впервые продемонстрирована полезность данных онлайн-ритейлеров для улучшения прогнозных свойств традиционных моделей прогнозирования инфляции.

     \begin{table}[h!]
     	\centering
     	\caption{Прогнозы инфляции}
     	\label{tab:inflation_forecasts}
     	\resizebox{\textwidth}{!}{%
     		\begin{tabular}{r|r|r|r|r|r|r}
     			\textbf{Онлайн} & \textbf{Офлайн} & \textbf{Средний Блумберг} & \textbf{Медиана Блумберг} & \textbf{AR(p)} & \textbf{Кривая Филлипса} & \textbf{RW} \\
     			\hline
     			0,429 & 0,806 & 1,677 & 1,785 & 1,198 & 1,968 & 1,138 \\
     		\end{tabular}
     	}
     \end{table}
     

Еще одним исследованием, посвященным использованию онлайн-данных по ценам для уточнения прогнозов по инфляции стала работа Национального банка Польши \cite{macias2023nowcasting}. Исследование является идейным продолжением работы \cite{aparicio2020forecasting}. Авторы фокусируются на предсказании продовольственной инфляции, поскольку доля продовольствия в потребительской корзине Польщи свыше 25\%, что является значительным вкладом в общий ИПЦ Польши.

Данные охватывают период с декабря 2009 г. по январь 2018 г. До 2016 года авторы собирали данные на еженедельной основе, а с 2016 г. – на ежедневной. Собранная база данных охватыват около 75 миллионов наблюдений цен на продовольственные товары, что покрывает примерно 488 тыс. товаров в 4-7 продуктовых онлайн-магазинах. Кроме того, для определения товаров по категориям была использована полуавтоматическая процедура разметки, в рамках которой сначала происходит разметка алгоритмами машинного обучения, затем производится ручной поиск ошибок разметки.

Для целей прогнозирования авторы агрегировали цены в месячную частоту. Авторы исследования использовали ту же модель, что и авторы \cite{aparicio2020forecasting}, однако без включения части, связанной с ценами на топливо. Производилась оценка 72 отдельных спецификаций (по одной на каждую продовольственную категорию). В качестве бенчмарков рассматривался авторегрессионный процесс (AR(p)), процесс скользящего среднего (RW), модель с сезонным случайным трендом, ARMA-модель и сезонная ARMA-модель. Результаты показали, что модель с онлайн-данными на коротком горизонте прогнозирования в среднем дает более точные прогнозы, чем в моделях без использования онлайн-данных.