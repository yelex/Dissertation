%&preformat-disser
\RequirePackage[l2tabu,orthodox]{nag} % Раскомментировав, можно в логе получать рекомендации относительно правильного использования пакетов и предупреждения об устаревших и нерекомендуемых пакетах
% Формат А4, 14pt (ГОСТ Р 7.0.11-2011, 5.3.6)
\documentclass[a4paper,14pt,oneside,openany]{memoir}
\DeclareUnicodeCharacter{2061}{} % для ликвидации ошибки U+2061 Unicode character ⁡ (U+2061) (inputenc) not set up for use with LaTeX
\DeclareUnicodeCharacter{2211}{}
\DeclareUnicodeCharacter{0302}{}
\DeclareUnicodeCharacter{03B1}{}
\DeclareUnicodeCharacter{03B2}{}


%%%%%%%%%%%%%%%%%%%%%%%%%%%%%%%%%%%%%%%%%%%%%%%%%%%%%%
%%%% Файл упрощённых настроек шаблона диссертации %%%%
%%%%%%%%%%%%%%%%%%%%%%%%%%%%%%%%%%%%%%%%%%%%%%%%%%%%%%

%%% Инициализирование переменных, не трогать!  %%%
\newcounter{intvl}
\newcounter{otstup}
\newcounter{contnumeq}
\newcounter{contnumfig}
\newcounter{contnumtab}
\newcounter{pgnum}
\newcounter{chapstyle}
\newcounter{headingdelim}
\newcounter{headingalign}
\newcounter{headingsize}
%%%%%%%%%%%%%%%%%%%%%%%%%%%%%%%%%%%%%%%%%%%%%%%%%%%%%%

%%% Область упрощённого управления оформлением %%%

%% Интервал между заголовками и между заголовком и текстом %%
% Заголовки отделяют от текста сверху и снизу
% тремя интервалами (ГОСТ Р 7.0.11-2011, 5.3.5)
\setcounter{intvl}{3}               % Коэффициент кратности к размеру шрифта

%% Отступы у заголовков в тексте %%
\setcounter{otstup}{0}              % 0 --- без отступа; 1 --- абзацный отступ

%% Нумерация формул, таблиц и рисунков %%
% Нумерация формул
\setcounter{contnumeq}{0}   % 0 --- пораздельно (во введении подряд,
                            %       без номера раздела);
                            % 1 --- сквозная нумерация по всей диссертации
% Нумерация рисунков
\setcounter{contnumfig}{0}  % 0 --- пораздельно (во введении подряд,
                            %       без номера раздела);
                            % 1 --- сквозная нумерация по всей диссертации
% Нумерация таблиц
\setcounter{contnumtab}{1}  % 0 --- пораздельно (во введении подряд,
                            %       без номера раздела);
                            % 1 --- сквозная нумерация по всей диссертации

%% Оглавление %%
\setcounter{pgnum}{1}       % 0 --- номера страниц никак не обозначены;
                            % 1 --- Стр. над номерами страниц (дважды
                            %       компилировать после изменения настройки)
\settocdepth{subsection}    % до какого уровня подразделов выносить в оглавление
\setsecnumdepth{subsection} % до какого уровня нумеровать подразделы


%% Текст и форматирование заголовков %%
\setcounter{chapstyle}{1}     % 0 --- разделы только под номером;
                              % 1 --- разделы с названием "Глава" перед номером
\setcounter{headingdelim}{1}  % 0 --- номер отделен пропуском в 1em или \quad;
                              % 1 --- номера разделов и приложений отделены
                              %       точкой с пробелом, подразделы пропуском
                              %       без точки;
                              % 2 --- номера разделов, подразделов и приложений
                              %       отделены точкой с пробелом.

%% Выравнивание заголовков в тексте %%
\setcounter{headingalign}{0}  % 0 --- по центру;
                              % 1 --- по левому краю

%% Размеры заголовков в тексте %%
\setcounter{headingsize}{0}   % 0 --- по ГОСТ, все всегда 14 пт;
                              % 1 --- пропорционально изменяющийся размер
                              %       в зависимости от базового шрифта

%% Подпись таблиц %%

% Смещение строк подписи после первой строки
\newcommand{\tabindent}{0cm}

% Тип форматирования заголовка таблицы:
% plain --- название и текст в одной строке
% split --- название и текст в разных строках
\newcommand{\tabformat}{plain}

%%% Настройки форматирования таблицы `plain`

% Выравнивание по центру подписи, состоящей из одной строки:
% true  --- выравнивать
% false --- не выравнивать
\newcommand{\tabsinglecenter}{false}

% Выравнивание подписи таблиц:
% justified   --- выравнивать как обычный текст («по ширине»)
% centering   --- выравнивать по центру
% centerlast  --- выравнивать по центру только последнюю строку
% centerfirst --- выравнивать по центру только первую строку (не рекомендуется)
% raggedleft  --- выравнивать по правому краю
% raggedright --- выравнивать по левому краю
\newcommand{\tabjust}{justified}

% Разделитель записи «Таблица #» и названия таблицы
\newcommand{\tablabelsep}{~\cyrdash\ }

%%% Настройки форматирования таблицы `split`

% Положение названия таблицы:
% \centering   --- выравнивать по центру
% \raggedleft  --- выравнивать по правому краю
% \raggedright --- выравнивать по левому краю
\newcommand{\splitformatlabel}{\raggedleft}

% Положение текста подписи:
% \centering   --- выравнивать по центру
% \raggedleft  --- выравнивать по правому краю
% \raggedright --- выравнивать по левому краю
\newcommand{\splitformattext}{\raggedright}

%% Подпись рисунков %%
%Разделитель записи «Рисунок #» и названия рисунка
\newcommand{\figlabelsep}{~\cyrdash\ }  % (ГОСТ 2.105, 4.3.1)
                                        % "--- здесь не работает

%%% Цвета гиперссылок %%%
% Latex color definitions: http://latexcolor.com/
%\definecolor{linkcolor}{rgb}{0.9,0,0}
%\definecolor{citecolor}{rgb}{0,0.6,0}
%\definecolor{urlcolor}{rgb}{0,0,1}
\definecolor{linkcolor}{rgb}{0,0,0} %black
\definecolor{citecolor}{rgb}{0,0,0} %black
\definecolor{urlcolor}{rgb}{0,0,0} %black
            % общие настройки шаблона
\input{common/packages}         % Пакеты общие для диссертации и автореферата
\synopsisfalse                      % Этот документ --- не автореферат
\input{Dissertation/dispackages}    % Пакеты для диссертации
\usepackage{fr-longtable}    %ради \endlasthead

% Листинги с исходным кодом программ
\usepackage{fancyvrb}
\usepackage{listings}
\lccode`\~=0\relax %Без этого хака из-за особенностей пакета listings перестают работать конструкции с \MakeLowercase и т. п. в (xe|lua)latex

% Русская традиция начертания греческих букв
\usepackage{upgreek} % прямые греческие ради русской традиции
\usepackage{tabularx} % для размещения таблицы в ширину страницы

%%% Микротипографика
%\ifnumequal{\value{draft}}{0}{% Только если у нас режим чистовика
%    \usepackage[final, babel, shrink=45]{microtype}[2016/05/14] % улучшает представление букв и слов в строках, может помочь при наличии отдельно висящих слов
%}{}

% Отметка о версии черновика на каждой странице
% Чтобы работало надо в своей локальной копии по инструкции
% https://www.ctan.org/pkg/gitinfo2 создать небходимые файлы в папке
% ./git/hooks
% If you’re familiar with tweaking git, you can probably work it out for
% yourself. If not, I suggest you follow these steps:
% 1. First, you need a git repository and working tree. For this example,
% let’s suppose that the root of the working tree is in ~/compsci
% 2. Copy the file post-xxx-sample.txt (which is in the same folder of
% your TEX distribution as this pdf) into the git hooks directory in your
% working copy. In our example case, you should end up with a file called
% ~/compsci/.git/hooks/post-checkout
% 3. If you’re using a unix-like system, don’t forget to make the file executable.
% Just how you do this is outside the scope of this manual, but one
% possible way is with commands such as this:
% chmod g+x post-checkout.
% 4. Test your setup with “git checkout master” (or another suitable branch
% name). This should generate copies of gitHeadInfo.gin in the directories
% you intended.
% 5. Now make two more copies of this file in the same directory (hooks),
% calling them post-commit and post-merge, and you’re done. As before,
% users of unix-like systems should ensure these files are marked as
% executable.
\ifnumequal{\value{draft}}{1}{% Черновик
   \IfFileExists{.git/gitHeadInfo.gin}{
      \usepackage[mark,pcount]{gitinfo2}
      \renewcommand{\gitMark}{rev.\gitAbbrevHash\quad\gitCommitterEmail\quad\gitAuthorIsoDate}
      \renewcommand{\gitMarkFormat}{\rmfamily\color{Gray}\small\bfseries}
   }{}
}{}   % Пакеты для специфических пользовательских задач

%%%%%%%%%%%%%%%%%%%%%%%%%%%%%%%%%%%%%%%%%%%%%%%%%%%%%%
%%%% Файл упрощённых настроек шаблона диссертации %%%%
%%%%%%%%%%%%%%%%%%%%%%%%%%%%%%%%%%%%%%%%%%%%%%%%%%%%%%

%%% Инициализирование переменных, не трогать!  %%%
\newcounter{intvl}
\newcounter{otstup}
\newcounter{contnumeq}
\newcounter{contnumfig}
\newcounter{contnumtab}
\newcounter{pgnum}
\newcounter{chapstyle}
\newcounter{headingdelim}
\newcounter{headingalign}
\newcounter{headingsize}
%%%%%%%%%%%%%%%%%%%%%%%%%%%%%%%%%%%%%%%%%%%%%%%%%%%%%%

%%% Область упрощённого управления оформлением %%%

%% Интервал между заголовками и между заголовком и текстом %%
% Заголовки отделяют от текста сверху и снизу
% тремя интервалами (ГОСТ Р 7.0.11-2011, 5.3.5)
\setcounter{intvl}{3}               % Коэффициент кратности к размеру шрифта

%% Отступы у заголовков в тексте %%
\setcounter{otstup}{0}              % 0 --- без отступа; 1 --- абзацный отступ

%% Нумерация формул, таблиц и рисунков %%
% Нумерация формул
\setcounter{contnumeq}{0}   % 0 --- пораздельно (во введении подряд,
                            %       без номера раздела);
                            % 1 --- сквозная нумерация по всей диссертации
% Нумерация рисунков
\setcounter{contnumfig}{0}  % 0 --- пораздельно (во введении подряд,
                            %       без номера раздела);
                            % 1 --- сквозная нумерация по всей диссертации
% Нумерация таблиц
\setcounter{contnumtab}{1}  % 0 --- пораздельно (во введении подряд,
                            %       без номера раздела);
                            % 1 --- сквозная нумерация по всей диссертации

%% Оглавление %%
\setcounter{pgnum}{1}       % 0 --- номера страниц никак не обозначены;
                            % 1 --- Стр. над номерами страниц (дважды
                            %       компилировать после изменения настройки)
\settocdepth{subsection}    % до какого уровня подразделов выносить в оглавление
\setsecnumdepth{subsection} % до какого уровня нумеровать подразделы


%% Текст и форматирование заголовков %%
\setcounter{chapstyle}{1}     % 0 --- разделы только под номером;
                              % 1 --- разделы с названием "Глава" перед номером
\setcounter{headingdelim}{1}  % 0 --- номер отделен пропуском в 1em или \quad;
                              % 1 --- номера разделов и приложений отделены
                              %       точкой с пробелом, подразделы пропуском
                              %       без точки;
                              % 2 --- номера разделов, подразделов и приложений
                              %       отделены точкой с пробелом.

%% Выравнивание заголовков в тексте %%
\setcounter{headingalign}{0}  % 0 --- по центру;
                              % 1 --- по левому краю

%% Размеры заголовков в тексте %%
\setcounter{headingsize}{0}   % 0 --- по ГОСТ, все всегда 14 пт;
                              % 1 --- пропорционально изменяющийся размер
                              %       в зависимости от базового шрифта

%% Подпись таблиц %%

% Смещение строк подписи после первой строки
\newcommand{\tabindent}{0cm}

% Тип форматирования заголовка таблицы:
% plain --- название и текст в одной строке
% split --- название и текст в разных строках
\newcommand{\tabformat}{plain}

%%% Настройки форматирования таблицы `plain`

% Выравнивание по центру подписи, состоящей из одной строки:
% true  --- выравнивать
% false --- не выравнивать
\newcommand{\tabsinglecenter}{false}

% Выравнивание подписи таблиц:
% justified   --- выравнивать как обычный текст («по ширине»)
% centering   --- выравнивать по центру
% centerlast  --- выравнивать по центру только последнюю строку
% centerfirst --- выравнивать по центру только первую строку (не рекомендуется)
% raggedleft  --- выравнивать по правому краю
% raggedright --- выравнивать по левому краю
\newcommand{\tabjust}{justified}

% Разделитель записи «Таблица #» и названия таблицы
\newcommand{\tablabelsep}{~\cyrdash\ }

%%% Настройки форматирования таблицы `split`

% Положение названия таблицы:
% \centering   --- выравнивать по центру
% \raggedleft  --- выравнивать по правому краю
% \raggedright --- выравнивать по левому краю
\newcommand{\splitformatlabel}{\raggedleft}

% Положение текста подписи:
% \centering   --- выравнивать по центру
% \raggedleft  --- выравнивать по правому краю
% \raggedright --- выравнивать по левому краю
\newcommand{\splitformattext}{\raggedright}

%% Подпись рисунков %%
%Разделитель записи «Рисунок #» и названия рисунка
\newcommand{\figlabelsep}{~\cyrdash\ }  % (ГОСТ 2.105, 4.3.1)
                                        % "--- здесь не работает

%%% Цвета гиперссылок %%%
% Latex color definitions: http://latexcolor.com/
%\definecolor{linkcolor}{rgb}{0.9,0,0}
%\definecolor{citecolor}{rgb}{0,0.6,0}
%\definecolor{urlcolor}{rgb}{0,0,1}
\definecolor{linkcolor}{rgb}{0,0,0} %black
\definecolor{citecolor}{rgb}{0,0,0} %black
\definecolor{urlcolor}{rgb}{0,0,0} %black
      % Упрощённые настройки шаблона

\input{common/newnames}         % Новые переменные, для всего проекта

%%% Основные сведения %%%
\newcommand{\thesisAuthorLastName}{\fixme{Евсеев}}
\newcommand{\thesisAuthorOtherNames}{\fixme{Алексей Сергеевич}}
\newcommand{\thesisAuthorInitials}{\fixme{А.\,С.}}
\newcommand{\thesisAuthor}             % Диссертация, ФИО автора
{%
    \texorpdfstring{% \texorpdfstring takes two arguments and uses the first for (La)TeX and the second for pdf
        \thesisAuthorLastName~\thesisAuthorOtherNames% так будет отображаться на титульном листе или в тексте, где будет использоваться переменная
    }{%
        \thesisAuthorLastName, \thesisAuthorOtherNames% эта запись для свойств pdf-файла. В таком виде, если pdf будет обработан программами для сбора библиографических сведений, будет правильно представлена фамилия.
    }
}
\newcommand{\thesisAuthorShort}        % Диссертация, ФИО автора инициалами
{\thesisAuthorInitials~\thesisAuthorLastName}
%\newcommand{\thesisUdk}                % Диссертация, УДК
%{\fixme{xxx.xxx}}
\newcommand{\thesisTitle}              % Диссертация, название
{\fixme{Использование онлайн данных для анализа ценовых тенденций в российской экономике}}
\newcommand{\thesisSpecialtyNumber}    % Диссертация, специальность, номер
{\fixme{5.2.1}}
\newcommand{\thesisSpecialtyTitle}     % Диссертация, специальность, название (название взято с сайта ВАК для примера)
{\fixme{Экономическая теория}}
%% \newcommand{\thesisSpecialtyTwoNumber} % Диссертация, вторая специальность, номер
%% {\fixme{XX.XX.XX}}
%% \newcommand{\thesisSpecialtyTwoTitle}  % Диссертация, вторая специальность, название
%% {\fixme{Теория и~методика физического воспитания, спортивной тренировки,
%% оздоровительной и~адаптивной физической культуры}}
\newcommand{\thesisDegree}             % Диссертация, ученая степень
{\fixme{кандидата экономических наук}}
\newcommand{\thesisDegreeShort}        % Диссертация, ученая степень, краткая запись
{\fixme{канд. экон. наук}}
\newcommand{\thesisCity}               % Диссертация, город написания диссертации
{\fixme{Москва}}
\newcommand{\thesisYear}               % Диссертация, год написания диссертации
{\the\year}
\newcommand{\thesisOrganization}       % Диссертация, организация
{\fixme{Федеральное государственное бюджетное образовательное учреждение высшего образования Российская академия народного хозяйства и государственной службы при Президенте Российской Федерации (РАНХиГС)}}
\newcommand{\thesisOrganizationShort}  % Диссертация, краткое название организации для доклада
{\fixme{НазУчДисРаб}}

\newcommand{\thesisInOrganization}     % Диссертация, организация в предложном падеже: Работа выполнена в ...
{\fixme{учреждении с~длинным длинным длинным длинным названием, в~котором
выполнялась данная диссертационная работа}}

%% \newcommand{\supervisorDead}{}           % Рисовать рамку вокруг фамилии
\newcommand{\supervisorFio}              % Научный руководитель, ФИО
{\fixme{Перевышин Юрий Николаевич}}
\newcommand{\supervisorRegalia}          % Научный руководитель, регалии
{\fixme{к. э. н.}}
\newcommand{\supervisorFioShort}         % Научный руководитель, ФИО
{\fixme{Ю.\,Н.~Перевышин}}
\newcommand{\supervisorRegaliaShort}     % Научный руководитель, регалии
{\fixme{к. э. н.}}

%% \newcommand{\supervisorTwoDead}{}        % Рисовать рамку вокруг фамилии
%% \newcommand{\supervisorTwoFio}           % Второй научный руководитель, ФИО
%% {\fixme{Фамилия Имя Отчество}}
%% \newcommand{\supervisorTwoRegalia}       % Второй научный руководитель, регалии
%% {\fixme{уч. степень, уч. звание}}
%% \newcommand{\supervisorTwoFioShort}      % Второй научный руководитель, ФИО
%% {\fixme{И.\,О.~Фамилия}}
%% \newcommand{\supervisorTwoRegaliaShort}  % Второй научный руководитель, регалии
%% {\fixme{уч.~ст.,~уч.~зв.}}

\newcommand{\opponentOneFio}           % Оппонент 1, ФИО
{\fixme{Фамилия Имя Отчество}}
\newcommand{\opponentOneRegalia}       % Оппонент 1, регалии
{\fixme{доктор физико-математических наук, профессор}}
\newcommand{\opponentOneJobPlace}      % Оппонент 1, место работы
{\fixme{Не очень длинное название для места работы}}
\newcommand{\opponentOneJobPost}       % Оппонент 1, должность
{\fixme{старший научный сотрудник}}

\newcommand{\opponentTwoFio}           % Оппонент 2, ФИО
{\fixme{Фамилия Имя Отчество}}
\newcommand{\opponentTwoRegalia}       % Оппонент 2, регалии
{\fixme{кандидат физико-математических наук}}
\newcommand{\opponentTwoJobPlace}      % Оппонент 2, место работы
{\fixme{Основное место работы c длинным длинным длинным длинным названием}}
\newcommand{\opponentTwoJobPost}       % Оппонент 2, должность
{\fixme{старший научный сотрудник}}

%% \newcommand{\opponentThreeFio}         % Оппонент 3, ФИО
%% {\fixme{Фамилия Имя Отчество}}
%% \newcommand{\opponentThreeRegalia}     % Оппонент 3, регалии
%% {\fixme{кандидат физико-математических наук}}
%% \newcommand{\opponentThreeJobPlace}    % Оппонент 3, место работы
%% {\fixme{Основное место работы c длинным длинным длинным длинным названием}}
%% \newcommand{\opponentThreeJobPost}     % Оппонент 3, должность
%% {\fixme{старший научный сотрудник}}

\newcommand{\leadingOrganizationTitle} % Ведущая организация, дополнительные строки. Удалить, чтобы не отображать в автореферате
{\fixme{Федеральное государственное бюджетное образовательное учреждение высшего
профессионального образования с~длинным длинным длинным длинным названием}}

\newcommand{\defenseDate}              % Защита, дата
{\fixme{DD mmmmmmmm YYYY~г.~в~XX часов}}
\newcommand{\defenseCouncilNumber}     % Защита, номер диссертационного совета
{\fixme{Д\,123.456.78}}
\newcommand{\defenseCouncilTitle}      % Защита, учреждение диссертационного совета
{\fixme{Название учреждения}}
\newcommand{\defenseCouncilAddress}    % Защита, адрес учреждение диссертационного совета
{\fixme{Адрес}}
\newcommand{\defenseCouncilPhone}      % Телефон для справок
{\fixme{+7~(0000)~00-00-00}}

\newcommand{\defenseSecretaryFio}      % Секретарь диссертационного совета, ФИО
{\fixme{Фамилия Имя Отчество}}
\newcommand{\defenseSecretaryRegalia}  % Секретарь диссертационного совета, регалии
{\fixme{д-р~физ.-мат. наук}}            % Для сокращений есть ГОСТы, например: ГОСТ Р 7.0.12-2011 + http://base.garant.ru/179724/#block_30000

\newcommand{\synopsisLibrary}          % Автореферат, название библиотеки
{\fixme{Название библиотеки}}
\newcommand{\synopsisDate}             % Автореферат, дата рассылки
{\fixme{DD mmmmmmmm}\the\year~года}

% To avoid conflict with beamer class use \providecommand
\providecommand{\keywords}%            % Ключевые слова для метаданных PDF диссертации и автореферата
{}
             % Основные сведения
\input{common/fonts}            % Определение шрифтов (частичное)
%%% Шаблон %%%
\DeclareRobustCommand{\fixme}{\textcolor{black}}  % решаем проблему превращения
                                % названия цвета в результате \MakeUppercase,
                                % http://tex.stackexchange.com/a/187930,
                                % \DeclareRobustCommand protects \fixme
                                % from expanding inside \MakeUppercase
\AtBeginDocument{%
    \setlength{\parindent}{2.5em}                   % Абзацный отступ. Должен быть одинаковым по всему тексту и равен пяти знакам (ГОСТ Р 7.0.11-2011, 5.3.7).
}

%%% Таблицы %%%
\DeclareCaptionLabelSeparator{tabsep}{\tablabelsep} % нумерация таблиц
\DeclareCaptionFormat{split}{\splitformatlabel#1\par\splitformattext#3}

\captionsetup[table]{
        format=\tabformat,                % формат подписи (plain|hang)
        font=normal,                      % нормальные размер, цвет, стиль шрифта
        skip=.0pt,                        % отбивка под подписью
        parskip=.0pt,                     % отбивка между параграфами подписи
        position=above,                   % положение подписи
        justification=\tabjust,           % центровка
        indent=\tabindent,                % смещение строк после первой
        labelsep=tabsep,                  % разделитель
        singlelinecheck=\tabsinglecenter, % не выравнивать по центру, если умещается в одну строку
}

%%% Рисунки %%%
\DeclareCaptionLabelSeparator{figsep}{\figlabelsep} % нумерация рисунков

\captionsetup[figure]{
        format=plain,                     % формат подписи (plain|hang)
        font=normal,                      % нормальные размер, цвет, стиль шрифта
        skip=.0pt,                        % отбивка под подписью
        parskip=.0pt,                     % отбивка между параграфами подписи
        position=below,                   % положение подписи
        singlelinecheck=true,             % выравнивание по центру, если умещается в одну строку
        justification=centerlast,         % центровка
        labelsep=figsep,                  % разделитель
}

%%% Подписи подрисунков %%%
\DeclareCaptionSubType{figure}
\renewcommand\thesubfigure{\asbuk{subfigure}} % нумерация подрисунков
\ifsynopsis
\DeclareCaptionFont{norm}{\fontsize{10pt}{11pt}\selectfont}
\newcommand{\subfigureskip}{2.pt}
\else
\DeclareCaptionFont{norm}{\fontsize{14pt}{16pt}\selectfont}
\newcommand{\subfigureskip}{0.pt}
\fi

\captionsetup[subfloat]{
        labelfont=norm,                 % нормальный размер подписей подрисунков
        textfont=norm,                  % нормальный размер подписей подрисунков
        labelsep=space,                 % разделитель
        labelformat=brace,              % одна скобка справа от номера
        justification=centering,        % центровка
        singlelinecheck=true,           % выравнивание по центру, если умещается в одну строку
        skip=\subfigureskip,            % отбивка над подписью
        parskip=.0pt,                   % отбивка между параграфами подписи
        position=below,                 % положение подписи
}

%%% Настройки ссылок на рисунки, таблицы и др. %%%
% команды \cref...format отвечают за форматирование при помощи команды \cref
% команды \labelcref...format отвечают за форматирование при помощи команды \labelcref

\ifpresentation
\else
    \crefdefaultlabelformat{#2#1#3}

    % Уравнение
    \crefformat{equation}{(#2#1#3)} % одиночная ссылка с приставкой
    \labelcrefformat{equation}{(#2#1#3)} % одиночная ссылка без приставки
    \crefrangeformat{equation}{(#3#1#4) \cyrdash~(#5#2#6)} % диапазон ссылок с приставкой
    \labelcrefrangeformat{equation}{(#3#1#4) \cyrdash~(#5#2#6)} % диапазон ссылок без приставки
    \crefmultiformat{equation}{(#2#1#3)}{ и~(#2#1#3)}{, (#2#1#3)}{ и~(#2#1#3)} % перечисление ссылок с приставкой
    \labelcrefmultiformat{equation}{(#2#1#3)}{ и~(#2#1#3)}{, (#2#1#3)}{ и~(#2#1#3)} % перечисление без приставки

    % Подуравнение
    \crefformat{subequation}{(#2#1#3)} % одиночная ссылка с приставкой
    \labelcrefformat{subequation}{(#2#1#3)} % одиночная ссылка без приставки
    \crefrangeformat{subequation}{(#3#1#4) \cyrdash~(#5#2#6)} % диапазон ссылок с приставкой
    \labelcrefrangeformat{subequation}{(#3#1#4) \cyrdash~(#5#2#6)} % диапазон ссылок без приставки
    \crefmultiformat{subequation}{(#2#1#3)}{ и~(#2#1#3)}{, (#2#1#3)}{ и~(#2#1#3)} % перечисление ссылок с приставкой
    \labelcrefmultiformat{subequation}{(#2#1#3)}{ и~(#2#1#3)}{, (#2#1#3)}{ и~(#2#1#3)} % перечисление без приставки

    % Глава
    \crefformat{chapter}{#2#1#3} % одиночная ссылка с приставкой
    \labelcrefformat{chapter}{#2#1#3} % одиночная ссылка без приставки
    \crefrangeformat{chapter}{#3#1#4 \cyrdash~#5#2#6} % диапазон ссылок с приставкой
    \labelcrefrangeformat{chapter}{#3#1#4 \cyrdash~#5#2#6} % диапазон ссылок без приставки
    \crefmultiformat{chapter}{#2#1#3}{ и~#2#1#3}{, #2#1#3}{ и~#2#1#3} % перечисление ссылок с приставкой
    \labelcrefmultiformat{chapter}{#2#1#3}{ и~#2#1#3}{, #2#1#3}{ и~#2#1#3} % перечисление без приставки

    % Параграф
    \crefformat{section}{#2#1#3} % одиночная ссылка с приставкой
    \labelcrefformat{section}{#2#1#3} % одиночная ссылка без приставки
    \crefrangeformat{section}{#3#1#4 \cyrdash~#5#2#6} % диапазон ссылок с приставкой
    \labelcrefrangeformat{section}{#3#1#4 \cyrdash~#5#2#6} % диапазон ссылок без приставки
    \crefmultiformat{section}{#2#1#3}{ и~#2#1#3}{, #2#1#3}{ и~#2#1#3} % перечисление ссылок с приставкой
    \labelcrefmultiformat{section}{#2#1#3}{ и~#2#1#3}{, #2#1#3}{ и~#2#1#3} % перечисление без приставки

    % Приложение
    \crefformat{appendix}{#2#1#3} % одиночная ссылка с приставкой
    \labelcrefformat{appendix}{#2#1#3} % одиночная ссылка без приставки
    \crefrangeformat{appendix}{#3#1#4 \cyrdash~#5#2#6} % диапазон ссылок с приставкой
    \labelcrefrangeformat{appendix}{#3#1#4 \cyrdash~#5#2#6} % диапазон ссылок без приставки
    \crefmultiformat{appendix}{#2#1#3}{ и~#2#1#3}{, #2#1#3}{ и~#2#1#3} % перечисление ссылок с приставкой
    \labelcrefmultiformat{appendix}{#2#1#3}{ и~#2#1#3}{, #2#1#3}{ и~#2#1#3} % перечисление без приставки

    % Рисунок
    \crefformat{figure}{#2#1#3} % одиночная ссылка с приставкой
    \labelcrefformat{figure}{#2#1#3} % одиночная ссылка без приставки
    \crefrangeformat{figure}{#3#1#4 \cyrdash~#5#2#6} % диапазон ссылок с приставкой
    \labelcrefrangeformat{figure}{#3#1#4 \cyrdash~#5#2#6} % диапазон ссылок без приставки
    \crefmultiformat{figure}{#2#1#3}{ и~#2#1#3}{, #2#1#3}{ и~#2#1#3} % перечисление ссылок с приставкой
    \labelcrefmultiformat{figure}{#2#1#3}{ и~#2#1#3}{, #2#1#3}{ и~#2#1#3} % перечисление без приставки

    % Таблица
    \crefformat{table}{#2#1#3} % одиночная ссылка с приставкой
    \labelcrefformat{table}{#2#1#3} % одиночная ссылка без приставки
    \crefrangeformat{table}{#3#1#4 \cyrdash~#5#2#6} % диапазон ссылок с приставкой
    \labelcrefrangeformat{table}{#3#1#4 \cyrdash~#5#2#6} % диапазон ссылок без приставки
    \crefmultiformat{table}{#2#1#3}{ и~#2#1#3}{, #2#1#3}{ и~#2#1#3} % перечисление ссылок с приставкой
    \labelcrefmultiformat{table}{#2#1#3}{ и~#2#1#3}{, #2#1#3}{ и~#2#1#3} % перечисление без приставки

    % Листинг
    \crefformat{lstlisting}{#2#1#3} % одиночная ссылка с приставкой
    \labelcrefformat{lstlisting}{#2#1#3} % одиночная ссылка без приставки
    \crefrangeformat{lstlisting}{#3#1#4 \cyrdash~#5#2#6} % диапазон ссылок с приставкой
    \labelcrefrangeformat{lstlisting}{#3#1#4 \cyrdash~#5#2#6} % диапазон ссылок без приставки
    \crefmultiformat{lstlisting}{#2#1#3}{ и~#2#1#3}{, #2#1#3}{ и~#2#1#3} % перечисление ссылок с приставкой
    \labelcrefmultiformat{lstlisting}{#2#1#3}{ и~#2#1#3}{, #2#1#3}{ и~#2#1#3} % перечисление без приставки

    % Листинг
    \crefformat{ListingEnv}{#2#1#3} % одиночная ссылка с приставкой
    \labelcrefformat{ListingEnv}{#2#1#3} % одиночная ссылка без приставки
    \crefrangeformat{ListingEnv}{#3#1#4 \cyrdash~#5#2#6} % диапазон ссылок с приставкой
    \labelcrefrangeformat{ListingEnv}{#3#1#4 \cyrdash~#5#2#6} % диапазон ссылок без приставки
    \crefmultiformat{ListingEnv}{#2#1#3}{ и~#2#1#3}{, #2#1#3}{ и~#2#1#3} % перечисление ссылок с приставкой
    \labelcrefmultiformat{ListingEnv}{#2#1#3}{ и~#2#1#3}{, #2#1#3}{ и~#2#1#3} % перечисление без приставки
\fi

%%% Настройки гиперссылок %%%
\ifluatex
    \hypersetup{
        unicode,                % Unicode encoded PDF strings
    }
\fi

\hypersetup{
    linktocpage=true,           % ссылки с номера страницы в оглавлении, списке таблиц и списке рисунков
%    linktoc=all,                % both the section and page part are links
%    pdfpagelabels=false,        % set PDF page labels (true|false)
    plainpages=false,           % Forces page anchors to be named by the Arabic form  of the page number, rather than the formatted form
    colorlinks,                 % ссылки отображаются раскрашенным текстом, а не раскрашенным прямоугольником, вокруг текста
    linkcolor={linkcolor},      % цвет ссылок типа ref, eqref и подобных
    citecolor={citecolor},      % цвет ссылок-цитат
    urlcolor={urlcolor},        % цвет гиперссылок
%    hidelinks,                  % Hide links (removing color and border)
    pdftitle={\thesisTitle},    % Заголовок
    pdfauthor={\thesisAuthor},  % Автор
    pdfsubject={\thesisSpecialtyNumber\ \thesisSpecialtyTitle},      % Тема
%    pdfcreator={Создатель},     % Создатель, Приложение
%    pdfproducer={Производитель},% Производитель, Производитель PDF
    pdfkeywords={\keywords},    % Ключевые слова
    pdflang={ru},
}
\ifnumequal{\value{draft}}{1}{% Черновик
    \hypersetup{
        draft,
    }
}{}

%%% Списки %%%
% Используем короткое тире (endash) для ненумерованных списков (ГОСТ 2.105-95, пункт 4.1.7, требует дефиса, но так лучше смотрится)
\renewcommand{\labelitemi}{\normalfont\bfseries{--}}

% Перечисление строчными буквами латинского алфавита (ГОСТ 2.105-95, 4.1.7)
%\renewcommand{\theenumi}{\alph{enumi}}
%\renewcommand{\labelenumi}{\theenumi)}

% Перечисление строчными буквами русского алфавита (ГОСТ 2.105-95, 4.1.7)
\makeatletter
\AddEnumerateCounter{\asbuk}{\russian@alph}{щ}      % Управляем списками/перечислениями через пакет enumitem, а он 'не знает' про asbuk, потому 'учим' его
\makeatother
%\renewcommand{\theenumi}{\asbuk{enumi}} %первый уровень нумерации
%\renewcommand{\labelenumi}{\theenumi)} %первый уровень нумерации
\renewcommand{\theenumii}{\asbuk{enumii}} %второй уровень нумерации
\renewcommand{\labelenumii}{\theenumii)} %второй уровень нумерации
\renewcommand{\theenumiii}{\arabic{enumiii}} %третий уровень нумерации
\renewcommand{\labelenumiii}{\theenumiii)} %третий уровень нумерации

\setlist{nosep,%                                    % Единый стиль для всех списков (пакет enumitem), без дополнительных интервалов.
    labelindent=\parindent,leftmargin=*%            % Каждый пункт, подпункт и перечисление записывают с абзацного отступа (ГОСТ 2.105-95, 4.1.8)
}

%%% Правильная нумерация приложений, рисунков и формул %%%
%% По ГОСТ 2.105, п. 4.3.8 Приложения обозначают заглавными буквами русского алфавита,
%% начиная с А, за исключением букв Ё, З, Й, О, Ч, Ь, Ы, Ъ.
%% Здесь также переделаны все нумерации русскими буквами.
\ifxetexorluatex
    \makeatletter
    \def\russian@Alph#1{\ifcase#1\or
       А\or Б\or В\or Г\or Д\or Е\or Ж\or
       И\or К\or Л\or М\or Н\or
       П\or Р\or С\or Т\or У\or Ф\or Х\or
       Ц\or Ш\or Щ\or Э\or Ю\or Я\else\xpg@ill@value{#1}{russian@Alph}\fi}
    \def\russian@alph#1{\ifcase#1\or
       а\or б\or в\or г\or д\or е\or ж\or
       и\or к\or л\or м\or н\or
       п\or р\or с\or т\or у\or ф\or х\or
       ц\or ш\or щ\or э\or ю\or я\else\xpg@ill@value{#1}{russian@alph}\fi}
    \def\cyr@Alph#1{\ifcase#1\or
        А\or Б\or В\or Г\or Д\or Е\or Ж\or
        И\or К\or Л\or М\or Н\or
        П\or Р\or С\or Т\or У\or Ф\or Х\or
        Ц\or Ш\or Щ\or Э\or Ю\or Я\else\xpg@ill@value{#1}{cyr@Alph}\fi}
    \def\cyr@alph#1{\ifcase#1\or
        а\or б\or в\or г\or д\or е\or ж\or
        и\or к\or л\or м\or н\or
        п\or р\or с\or т\or у\or ф\or х\or
        ц\or ш\or щ\or э\or ю\or я\else\xpg@ill@value{#1}{cyr@alph}\fi}
    \makeatother
\else
    \makeatletter
    \if@uni@ode
      \def\russian@Alph#1{\ifcase#1\or
        А\or Б\or В\or Г\or Д\or Е\or Ж\or
        И\or К\or Л\or М\or Н\or
        П\or Р\or С\or Т\or У\or Ф\or Х\or
        Ц\or Ш\or Щ\or Э\or Ю\or Я\else\@ctrerr\fi}
    \else
      \def\russian@Alph#1{\ifcase#1\or
        \CYRA\or\CYRB\or\CYRV\or\CYRG\or\CYRD\or\CYRE\or\CYRZH\or
        \CYRI\or\CYRK\or\CYRL\or\CYRM\or\CYRN\or
        \CYRP\or\CYRR\or\CYRS\or\CYRT\or\CYRU\or\CYRF\or\CYRH\or
        \CYRC\or\CYRSH\or\CYRSHCH\or\CYREREV\or\CYRYU\or
        \CYRYA\else\@ctrerr\fi}
    \fi
    \if@uni@ode
      \def\russian@alph#1{\ifcase#1\or
        а\or б\or в\or г\or д\or е\or ж\or
        и\or к\or л\or м\or н\or
        п\or р\or с\or т\or у\or ф\or х\or
        ц\or ш\or щ\or э\or ю\or я\else\@ctrerr\fi}
    \else
      \def\russian@alph#1{\ifcase#1\or
        \cyra\or\cyrb\or\cyrv\or\cyrg\or\cyrd\or\cyre\or\cyrzh\or
        \cyri\or\cyrk\or\cyrl\or\cyrm\or\cyrn\or
        \cyrp\or\cyrr\or\cyrs\or\cyrt\or\cyru\or\cyrf\or\cyrh\or
        \cyrc\or\cyrsh\or\cyrshch\or\cyrerev\or\cyryu\or
        \cyrya\else\@ctrerr\fi}
    \fi
    \makeatother
\fi


%%http://www.linux.org.ru/forum/general/6993203#comment-6994589 (используется totcount)
\makeatletter
\def\formtotal#1#2#3#4#5{%
    \newcount\@c
    \@c\totvalue{#1}\relax
    \newcount\@last
    \newcount\@pnul
    \@last\@c\relax
    \divide\@last 10
    \@pnul\@last\relax
    \divide\@pnul 10
    \multiply\@pnul-10
    \advance\@pnul\@last
    \multiply\@last-10
    \advance\@last\@c
    #2%
    \ifnum\@pnul=1#5\else%
    \ifcase\@last#5\or#3\or#4\or#4\or#4\else#5\fi
    \fi
}
\makeatother

\newcommand{\formbytotal}[5]{\total{#1}~\formtotal{#1}{#2}{#3}{#4}{#5}}

%%% Команды рецензирования %%%
\ifboolexpr{ (test {\ifnumequal{\value{draft}}{1}}) or (test {\ifnumequal{\value{showmarkup}}{1}})}{
        \newrobustcmd{\todo}[1]{\textcolor{red}{#1}}
        \newrobustcmd{\note}[2][]{\ifstrempty{#1}{#2}{\textcolor{#1}{#2}}}
        \newenvironment{commentbox}[1][]%
        {\ifstrempty{#1}{}{\color{#1}}}%
        {}
}{
        \newrobustcmd{\todo}[1]{}
        \newrobustcmd{\note}[2][]{}
        \excludecomment{commentbox}
}
           % Стили общие для диссертации и автореферата
\input{Dissertation/disstyles}  % Стили для диссертации
\input{Dissertation/userstyles} % Стили для специфических пользовательских задач

%%% Библиография. Выбор движка для реализации %%%
% Здесь только проверка установленного ключа. Сама настройка выбора движка
% размещена в common/setup.tex
\ifnumequal{\value{bibliosel}}{0}{%
    \input{biblio/predefined}   % Встроенная реализация с загрузкой файла через движок bibtex8
}{
    \input{biblio/biblatex}     % Реализация пакетом biblatex через движок biber
}

% Вывести информацию о выбранных опциях в лог сборки
\typeout{Selected options:}
\typeout{Draft mode: \arabic{draft}}
\typeout{Font: \arabic{fontfamily}}
\typeout{AltFont: \arabic{usealtfont}}
\typeout{Bibliography backend: \arabic{bibliosel}}
\typeout{Precompile images: \arabic{imgprecompile}}
% Вывести информацию о версиях используемых библиотек в лог сборки
\listfiles

%%% Управление компиляцией отдельных частей диссертации %%%
% Необходимо сначала иметь полностью скомпилированный документ, чтобы все
% промежуточные файлы были в наличии
% Затем\part{title}, для вывода отдельных частей можно воспользоваться командой \includeonly
% Ниже примеры использования команды:
%
%\includeonly{Dissertation/part2}
%\includeonly{Dissertation/contents,Dissertation/appendix,Dissertation/conclusion}
%
% Если все команды закомментированы, то документ будет выведен в PDF файл полностью

\begin{document}
%%% Переопределение именований типовых разделов
% https://tex.stackexchange.com/a/156050
\gappto\captionsrussian{\input{common/renames}\unskip} % for polyglossia and babel
\input{common/renames}

%%% Структура диссертации (ГОСТ Р 7.0.11-2011, 4)
\include{Dissertation/title}           % Титульный лист
\include{Dissertation/contents}        % Оглавление
\ifnumequal{\value{contnumfig}}{1}{}{\counterwithout{figure}{chapter}}
\ifnumequal{\value{contnumtab}}{1}{}{\counterwithout{table}{chapter}}
\include{Dissertation/introduction}    % Введение
\ifnumequal{\value{contnumfig}}{1}{\counterwithout{figure}{chapter}
}{\counterwithin{figure}{chapter}}
\ifnumequal{\value{contnumtab}}{1}{\counterwithout{table}{chapter}
}{\counterwithin{table}{chapter}}
\chapter{Обзор эмпирических исследований, посвященных изучению динамики цен онлайн-ритейлеров}\label{ch:ch1}

\section{Эволюция исследований по использованию онлайн-данных в изучении отдельных аспектов ценовой динамики}\label{sec:ch1/sec1}

Одной из первых работ, посвященных изучению поведения цен интернет-ритейлеров, стало исследование \cite{RePEc:ecb:ecbwps:2006645}. Основная цель исследования состояла в том, чтобы оценить различие между степенью жесткости цен в интернет-магазинах и традиционных (т.е. офлайн-) магазинах, а также получить оценки жестоксти для различных стран, типов товаров, и определить, от каких факторов зависит жесткость цен.

Авторы использовали уникальную базу данных по 5 миллионам ценовых котировок, собранных с сайтов-агрегаторов 4-х европейских стран (Франция, Италия, Германия и Великобритания) и США за период с декабря 2004 года по декабрь 2005 года. Данные скачивались на ежедневной основе. На момент сбора данных интернет-торговля была не слишком широко развита и покрывала в основном непродовольственные товары. Набор данных, собранных авторами, покрывал потребительскую электронику (DVD-плееры, телевизоры, домашние мини-системы), развлекательную электронику (портативные mp3-плееры, цифровые видеокамеры), компьютерную технику (ноутбуки, сканеры), кухонную технику (микроволновые печи, кофемашины), мелкую бытовую технику (пылесосы), крупную бытовую технику (холодильники, стиральные машины) и услуги (фотопроявка). Использование фотопроявки в качестве примера конкретной услуги было продиктовано тем, что услуги в Европе на тот момент были в целом слабо представлены в интернете, и кроме того, фотопроявка являлась достаточно четко определенной категорией услуг, что позволяет сравнивать свойства жесткости цен по ней между различными странами. Практически половина от собранных авторами данных (48\%) приходится на данные из США.

Далее в работе описываются стилизованные факты относительно жесткости цен в Европе и США. Авторы отмечают, что несмотря на низкие издержки изменения, цены не менялись на ежедневной основе. Последнее наблюдалось для всех стран и категорий товаров. В среднем на всей выборке (т.е. по всем категориям и рассматриваемым странам) ежедневная частота изменений цен составила 2,6\% (т.е. 2,6\% ценовых котировок менялись ежедневно). На уровне стран наибольшая частота изменений цен наблюдалась в Италии (около 3,8\%), а наименьшеая – в Великобритании (2,1\%). Средняя частота изменений цен во Франции составила 3,1\%, в Германии – 2,7\%, в США – 2,5\%. Для большей интерпретируемости результатов авторы использовали понятие дюрации, или среднего периода неизменности цен ($D^{av}_{k}$), который рассчитывается, отталкиваясь от средней частоты ценовых изменений:

\begin{equation}
	\label{eq:equation1}
	D_k^{av}=\frac{-1}{ln⁡(1-F_k)},
\end{equation}
где \( k \) "--- идентификтор страны или категории товара, \( F_k \) "--- средняя частота изменения.

Авторы получили разброс оценок средней дюрации от 25 дней во Франции до 68 дней – в США. В среднем для 4-х европейских стран авторы получили оценку в 31 день – то есть цены по используемым в работе данным остаются неизменными в среднем месяц. Авторы отмечают, что эти результаты противоречат оценкам по США на традиционных данных [2], где был получен период неизменности в 6-7 месяцев. Вероятно, более низкая оценка жесткости, полученная авторами, является следствием в целом более высокой частоты изменений в интернете и использованием данных дневной, а не месячной частоты.

На уровне отдельных товаров также наблюдается гетерогенность – так, минимальная средняя частота изменений варьировалась от 1,3\% в день для кофемашин до 4,3\% - для ноутбуков. После кофемашин, наименьшей средней частотой характеризовались микроволновые печи (1,5\% в день), домашние мини-системы (1,5\%), пылесосы (1,6\%) и холодильники (1,6\%). Авторы предполагают, что низкая частота изменений цен на эти категории товаров продиктована относительно более длительным сроком службы и меньшей скоростью морального устаревания по сравнению с остальными товарами, чьи цены в среднем меняются чаще (например, ноутбуками). Стоит также отметить, что среди всех сочетаний продуктовых категорий-исследуемых стран наименьшей частотой изменения характеризовались микроволновые печи в США (0,4\% в день), а наибольшей – телевизоры в Великобритании (6,3\%).

Авторы отдельно отмечают категорию фотопроявки. За весь рассматриваемый период из 13000 ценовых котировок, собранных для европейских стран, было зафиксировано лишь 14 изменений цен. Этот результат может свидетельствовать о том, что торговля в онлайн-ритейле приводит к повышению гибкости цен в сфере товаров, но не услуг. Сфера услуг в целом отличается более жесткими ценами по сравнению с обычными потребительскими товарами, что было показано, к примеру, на данных США \cite{bils2004some}.

В работе также отмечается, что гетерогенность в жесткости цен между странами менее выражена по сравнению с гетерогенностью между отдельными категориями товаров и услуг. Так, средняя частота изменений цен по категориям, как было отмечено выше, варьируется от 1,3\% (для кофемашин) до 4,3\% (для ноутбуков) в день, в то время как между странами средняя частота варьируется от 3,4\% (для Франции) до 5,6\% (для Великобритании). Нужно сказать, что разница в частотах и на отдельные категории товаров (за исключением микроволновых печей, холодильников и пылесосов) и услуг не сильно варьируется между отдельными странами. 

Авторы отмечают, что снижения цен в интернете случаются чаще, чем было найдено по данным европейских офлайн-цен \cite{dhyne2005price}. В работе \cite{dhyne2005price} отмечалось, что лишь 4 из 10 ценовых изменений являются отрицательными. Авторы работы \cite{RePEc:ecb:ecbwps:2006645} на данных онлайн-цен обнаружили большую долю снижений: от 40\% для микроволновых печей в Великобритании до 87\% на телевизоры (также в Великобритании). В среднем по всем странам и категориям, доля снижений цен составила 62\%. Вместе с тем авторы отмечают, что столь высокая доля снижений может объясняться особенностями товаров в используемой выборке: поскольку основная доля товаров – это электроника с высоким темпом морального устаревания – то частое снижение цен перед введением новой модели является широко распространенным явлением, увеличивающим долю снижений цен на общем фоне.

Авторами также была обнаружена высокая степень гетерогенности между типами магазинов. Так, компании, которые занимаются доставкой товаров по почте, а также ТВ-магазины имеют в среднем наименьшую частоту изменений цен. Этот результат согласуется с концепцией издержек меню, поскольку эти два типа магазинов несут наибольшие издержки при изменении цен по сравнению с остальными типами магазинов. Тот же аргумент объясняет, почему магазины, продающие товары только онлайн, имеют наибольшую частоту изменений цен.

В работе было также обнаружено, что средний размер изменений цен является относительно высоким и составляет 5,4\%. Вместе с тем, как отмечают авторы, это число меньше, чем оценки, полученные на традиционных офлайн-данных. Интервал, в котором происходило большинство изменений цен в онлайн-ритейле, составляет от 0 до -1\%, что разнится с данными традиционных офлайн-ритейлеров (у которых лишь малая доля изменений находится в промежутке от -2,5\% до 2,5\% \cite{Hoffmann2006}). Последнее также соотносится с предсказаниями концепции издержек меню, поскольку издержки изменения цен у онлайн-ритейлеров ниже, эти изменения для них дешевле, и потому они могут корректировать их чаще.

Наконец, авторы построили панельную логит-модель, в которой изменения цен попытались объяснить как факторами, зависящими от времени (time-dependent variables), так и факторами, зависящими от рыночной обстановки (state-dependent variables). 

Результаты модели показали, что частота изменений цен увеличивается с увеличением числа продавцов, предлагающих этот продукт, а также с увеличением доли изменений цен на этот товар, произошедших в предыдущий день. Привлекательные цены (такие как €9,99), а также относительно высокие цены снижают вероятность их изменения при прочих равных. В целом, эти результаты оказываются устойчивыми для всех подвыборок товаров, продающихся во всех рассматриваемых странах.

Ключевым в области изучения поведения цен онлайн-ритейлеров стал проект The Billion Prices Project. Проект значительно отличается от предыдущих попыток сбора данных по ценам онлайн-ритейлеров, поскольку является более широкомасштабным, методологически совершенным и устойчивым во времени. Подробное описание проекта приводится в работе \cite{cavallo2016billion}.

Авторы проекта отмечают, что его появление было мотивировано манипулированием данных по инфляции в Аргентине с 2007 по 2015 гг. К 2007 году стало понятно, что официально публикуемый уровень инфляции в Аргентине значительно отличается от того, что на самом деле происходит с ценами: это показывали как расчеты местных экономистов, так и опросы домохозяйств. Авторы будущего проекта стали собирать данные по ценам на ежедневной основе и показали, что на фоне официально заявляемой ежегодной инфляцией за 2007-2011 гг. в 8\% данные по ценам онлайн-ритейлеров демонстировали среднюю ежегодную инфляцию в 20\%. Авторы на данных онлайн-ритейлеров также показали, что манипуляция с официально публикуемой инфляцией завершилась в декабре 2015 года, с избранием нового правительства в Аргентине. Таким образом, случай с обнаружением значительных статистических расхождений между онлайн- и офлайн-инфляцией показал, что данные по ценам онлайн-ритейлеров обладают существенным потенциалом для использования в измерении инфляции.

Последнее привело к созданию The Billion Prices Project в Массачусетском технологическом институте – проекта по сбору цен онлайн-ритейлеров для нескольких стран, включая США. К 2010 году в проекте уже собиралось около 5 миллионов цен порядка 300 ритейлеров из 50 стран мира. Несмотря на то, что собирать цены в интернете значительно дешевле, чем традиционным офлайн-способом, проект столкнулся с проблемами финансирования, что привело к созданию коммерческого ответвления PriceStats – компании, которая предоставляет данные по высокочастотным индексам для центральных банков и клиентов финансового сектора.

Авторы отмечают важность методологии сбора данных. Авторы тщательно отбирают ритейлеров, используемых как источники данных, используют технологии «веб-скраппинга» для сбора данных, затем производят очистку и приводят данные в соответствие с целями исследований или измерения инфляции. 

Стоит остановиться на первом этапе методологии. Несмотря на огромный массив данных, критически важным для целей измерения и прогнозирования инфляции является тщательный отбор как категорий, так и ритейлеров. Ключевой целью авторов проекта является репрезентативный сбор транзакций. При отборе ритейлеров авторы стараются игнорировать ритейлеров, которые продают свои товары исключительно онлайн, и сосредотачиваются в основном на мультиканальных ритейлерах – то есть ритейлерах, которые продают товары как через интернет, так и традиционным офлайн-способом (речь идет о таких магазинах, как Walmart). Как поясняют авторы проекта, главная причина такого внимания к мультиканальным ритейлерам состоит в том, что они в подавляющем большинстве стран мира являются вовлеченными в большинство транзакций, что важно с точки зрения репрезентативности ценовых индексов. Авторы также отмечают, что при сборе данных по ценам внутри таких ритейлеров они сосредотачиваются, как правило, на тех категориях товаров, которые являются частью официальной корзины национального индекса потребительских цен, и стараются избежать товаров, которые чрезмерно представлены в онлайн-ритейле (речь о таких товарах, как CD-, DVD-диски, косметика и книги).

После сбора данных авторы приступают к их очистке, стандартизации для соответствия общей схеме базы данных, классификации отдельных товаров по категориям индекса потребительских цен и расчету простых характеристик. Каждый из ритейлеров является уникальной «стратой» с уникальными характеристиками и ценовым поведением. Перед тем, как включить ритейлера в процесс сбора данных для расчета индекса цен, авторы мониторят поведение ритейлера в течение года для идентификации любых специфичных характеристик в собираемых данных чтобы понять, насколько в целом будет полезным включение данного ритейлера для расчета ценовых индексов.

Авторы отмечают, что объем данных и покрытие различных категорий отличается между странами. Для приблизительно 25 стран собираемые авторами данные покрывают как минимум 70\% весов локальных индексов потребительских цен.

Следующим после отбора источников данных шагом является процесс непосредственного сбора данных ценовой информации. Авторы используют технологию «веб-скраппинга», которая с течением времени значительно улучшилась. Если раньше «веб-скраппинг» требовал от исследователей написания программ на таких языках как Python или PHP, то сегодня существует много «point-and-click» систем, позволяющих без специальных навыков программирования обучить программу на сбор данных с определенных частей страницы. Такое программное обеспечение позволяет создать робота, который будет способен извлекать нужную информацию из веб-сайта с однородной структурой и помещать эту информацию в базу данных. Вызовом для сбора данных является обнаружение ошибок, возникающих с течением времени (например, из-за изменения разметки сайта). Авторы собирают следующие данные: название товара, его описание, бренд, размер, информацию о категории и цену (если доступно, то еще информацию о том, отсутствует ли товар и является ли цена распродажной).

Отдельно авторы сосредоточились на преимуществах и недостатках собираемых данных. Авторы производили сравнение с традиционными данными, лежащими в основе расчета национального индекса потребительских цен, а также данными, собираемыми независимыми агентствами, такими как AC Nielsen. 

Одним из главных преимуществ онлайн-данных по ценам является низкая стоимость наблюдений. Как отмечают авторы, издержки сбора хотя и не тривиальны, однако являются гораздо более низкими, чем оплата труда сотрудников, которые будут физически посещать магазины, или стоимость наблюдений у таких провайдеров как AC Nielsen.

Еще одним достоинством онлайн-данных является высокая частота наблюдений - чаще всего, дневная, однако существует возможность собирать данные с любой возможной частотой. Также это преимущество позволяет избежать усреднения по времен, что является частым явлением в сборе традиционных офлайн-данных по ценам.

Третьим важным преимуществом является наличие детализированной информации для всех товаров в выборке ритейлеров. Как правило, это преиущство еще и дополняется большим объемом ценовых котировок, собираемых внутри категорий, чем в случае традиционных данных (в этом случае собирается, как правило, 5-10 котировок на категорию). Такое преимущество позволяет избежать проблему оценки изменения качества при исчезновении одного товара и появлении другого.

В-четвертых, данные онлайн-ритейлеров не имеют цензурированных рядов цен. Цены на товары собираются до тех пор, пока товар не исчезнет из магазина. Традиционные методы напротив, часто начинают наблюдать новые товары только тогда, когда исчезнет предыдущие, и таким образом информация о более ранней ценовой истории товара не наблюдается.

Пятое преимущество состоит в том, что данные онлайн-ритейлеров собираются удаленно. Авторы указывают, что такое преимущество становится особенно ценным как в случае с Аргентиной, в которой правительство пыталось всячески помешать независимому сбору данных для расчета инфляции. Это преимущество также позволяет сделать сбор данных централизованным и более однородным с точки зрения характеристик собираемых данных.

Шестое преимущество следует из пятого и состоит в том, что наборы собранных данных могут быть прямо сопоставлены между отдельными странами, поскольку методы сбора данных для разных стран применяются одни и те же. Это особенно полезно в исследовательских ценлях сопоставления между отдельными странами, товарами и временными периодами.

Наконец, данные онлайн-ритейлеров являются доступными в режиме реального времени, с отсутствием какой-либо задержки в доступе и обработке информации. Это преимущество особенно важно для лиц, принимающих оперативные решения в области денежно-кредитной политики и для всех специалистов, нуждающихся в максимально оперативной информации о ценах.

Одним из основных недостатков собираемых данных является гораздо меньшее покрытие ритейлеров и продуктовых категорий, чем в случае собираемой национальными органами статистики. В частности, цены большинства услуг все еще остаются недоступными в интернете, и кроме того число и разнообразие ритейлеров остается ограниченным по сравнению с данными официальной статистики.

Еще одним недостатком онлайн-данных по ценам является отсутствие информации о количестве проданных товаров. До сих пор данные онлайн-ритейлеров сочетаются с весами официальной статистики для использования в расчетах, требующих этих весов. Авторы отмечают, что данные независимых агентств, таких как AC Nielsen, напротив, имеют детализированную информацию о количестве проданных товаров, и потому могут быть потенциально использованы как источник высокочастотных данных о весах в некоторых категориях товаров, например в продовольствии.

Авторы отдельно останавливаются на вопросе о том, являются ли цены онлайн- и офлайн- различными. Этот вопрос является важным, поскольку распространение выводов из онлайн- на офлайн-торговлю требует предпосылки, что цены в целом ведут себя одинаково (а для некоторых выводов требуется также, чтобы и уровни цен совпадали). В работах \cite{brynjolfsson2000frictionless, ellison2009search, gorodnichenko2018price} авторы показали, что в онлайн-ритейле цены, по-видимому, меняются чаще и на меньшую величину, чем цены, лежащие в основе расчета индекса потребительских цен. Однако, ритейлеры, которые использовались в вышеупомянутых исследованиях, как правило продавали свою продукцию только через интернет, что отличается от используемых в работе данных по мультиканальным ритейлерам.

Чтобы понять, насколько отличаются цены и их динамика в онлайн- и офлайн-ритейле для мультиканальных ритейлеров, в работе \cite{cavallo2016online} были описаны данные по ценам на 24000 товаров, которые собирались одновременно онлайн- и офлайн в 56 крупнейших ритейлерах 10 стран. Сопоставление было реализовано путем создания специального программного обеспечения для регистрации цен в физических магазинах, привлечения волонтеров и техник «веб-скраппинга». Прямое сопоставление показало как высокую степень схожести в уровнях цен, так и в частоте и размере изменений цен. Результаты показали, что 70\% уровней цен на один и тот же товар схожи в онлайн- и офлайн-ритейле. Изменения цен хотя и не были точно синхронизированны, однако размер и частота изменений в целом были схожи между собой. Стоит заметить, что отсутствие синхронизации может в целом приводить к тому, что изменения онлайн-цен, вероятно, могут предсказывать изменения цен в офлайн-рителе.

Часть исследования \cite{cavallo2016billion} была посвящена изучению построения индексов инфляции на данных онлайн-ритейлеров. Авторы на примере нескольких латиноамериканских стран показывают, что данные по ценам онлайн-ритейлеров могут быть качественным альтернативным источником для построения индексов, которые будут демонстрировать близкую динамику к официальным индексам цен. Различие возникает в основном в уровне инфляции, но не в динамики индекса с течением времени. Авторы также показывают, что онлайн-индекс значительно быстрее, чем официальный, реагирует на агрегированные шоки.

Кроме того, синхронность онлайн- и офлайн-индексов инфляции для разных стран оказывается различным. Авторы показывают, что для США эти индексы демонстрируют очень близкую между собой динамику в течение 7 лет наблюдений. Столь высокая степень сходства объясняется относительно высокой (по сравнению с другими странами) долей интернет-торговли в общем объеме розничной торговли США, а также высокой долей мультиканальных ритейлеров в общем торговом обороте. В целом индексы, построенные на данных онлайн-ритейлеров, достаточно близко реплицируют поведение официальной инфляции, что справедливо как для крупных, так и для малых стран, для развитых и для развивающихся рынков.

Авторы отдельно сосредотачивают внимание на возможности онлайн-индекса цен предсказывать будущее направление и амплитуду изменений в офлайн-инфляции после того или иного шока. Так, в 2008 году, после банкротства Lehman Brother’s в США онлайн-индекс продемонстрировал драматическое падение практически сразу, в то время как официальному индексу потребовалось 2 месяца, чтобы показать тенденцию инфляции к замедлению.

Важной проблемой для сбора данных на длинную перспективу является т.н. «перекрытие качества» (overlapping quality). Проблема состоит в том, что при исчезновении того или иного товара из продажи его требуется заменить на новый товар, однако для сохранения сопоставимости ряда во времени нужно нивелировать различие в качестве между старым и новым товарами. Поскольку возможности статистических органов по сбору данных не такие широкие, как в случае сбора данных онлайн-ритейлеров, то на каждую категорию товаров им приходится собирать сильно ограниченное количество товаров или услуг. В большинстве национальных статистических ведомств используются различные техники для аппроксимации динамики в момент перехода от одного товара к другому, которые помогают сгладить различия в качестве заменяемых товаров. Часто для этих целей используются гедонистические регрессии, однако зачастую они имеют ряд ограничений, связанных с оценкой качества товаров. Данные онлайн-ритейлеров в этих ситуациях имеют преимущество, поскольку из-за большого числа собираемых марок/моделей/уникальных товаров в рамках каждой из категорий исчезновение одного из товаров не сильно влияет на общую динамику цен. 

Как было отмечено раннее, онлайн-индекс полезен для предсказания официальной инфляции. Для документирования этого факта авторы построили авторегрессионную модель с распределенным лагом, где в качестве зависимой переменной был использован официальным ИПЦ США и онлайн-индекс цен в качестве независимой переменной, и рассчитали импульсные отклики, чтобы посмотреть, как шоки онлайн-индека влияют на офлайн-индекс с течением времени. Были использованы ежемесячные разности как в случае зависимой, так и независимой переменной, а также 6 лагов запаздывания каждой из переменных. Результаты показали, что для США традиционный офлайн-индекс требует нескольких месяцев для учета информации из изменения онлайн-индекса. На уровне отдельных секторов наиболее быстрый эффект наблюдается в области цен на топливо и наиболее медленный – в сфере продуктов и электроники. Авторы подчеркивают, что способность онлайн-индексов предсказывать поведение офлайн-индекса может объясняться задержкой в публикации данных традиционных индексов, различиями в выборках магазинов, а также более быстрой адаптацией цен в онлайн-ритейле для некоторых отдельных секторов. 

Еще одним приложением данных по ценам онлайн-ритейлеров является уточнение выводов и оценок по жесткости цен. Последнее является фундаментальным элементом многих макроэкономических моделей. В последние десятилетия появился большой объем эмпирической литературы, посвященной оценке тех или иных аспектов поведения цен на микроуровне и изучению оснований этого поведения (см., к примеру, \cite{bils2004some,klenow2008state,gagnon2009}). Эти исследования стали возможны благодаря бесперецедентному доступу исследователей к базам данных по ценам, лежащим в основе расчета национальных индексов потребительских цен.

Вместе с тем, работа \cite{cavallo2018scraped} на данных онлайн-ритейлеров показала, что полученные выводы являются смещенными из-за характеристик данных по ценам, используемым для построения индексов потребительских цен, а также данных независимых агентств. В частности, автор показал, что два свойства этих данных ведут к смещению: усреднение цен во времени (характерно для данных независимых агентств) и специфическая для данных по ценам официальных статистических ведомств замена пропущенных цен. Автор подчеркивает, что онлайн-данные по ценам лишены подобных проблем, поскольку однажды настроенная программа по сбору цен собирает их на ежедневной основе (таким образом, отсутстве усреднение в конце недели) и без замены пропущенных цен какими-либо методами (как в случае данных национальных статистических ведомств).

Используемый авторами онлайн-датасет включает в себя более 60 миллионов ежедневных наблюдений по ценам в пяти странах: Аргентина, Бразилия, Чили, Колумбия и США. Данные были собраны с сайтов 8 разных компаний за период с 2007 по 2010 гг. Для США использовались данные 4 крупнейших ритейлеров в стране: супермаркет, гипермаркет/универмаг, аптека и ритейлер, который продает в основном электронику. В других странах использовались данные крупнейшего ритейлера в стране. Все эти ритейлеры являются лидерами в своих странах с рыночной долей около 28\% в Аргентине, 15\% в Бразилии, 27\% в Чили и 30\% в Колумбии. Авторы отмечают, что в данных встречается большое количество пропущенных наблюдений (в основном вследствие ошибок в «веб-скраппинге» или временном отсутствии товара на складе). В зависимости от страны, доля пропущенных значений варьируется от 22\% до 37\% от всех наблюдений. Часть изменений цен оказалась слишком большой по амплитуде, что является следствием ошибок в процессе сбора данных. По этой причине из данных были удалены изменения свыше 200\% и меньше -70\%.

Чтобы показать эффект усреднения, который применяется в данных агентства Nielsen, автор сопоставляет динамику цен для одного и того же ритейлера, его местоположения и временного периода онлайн- и офлайн-. Автор на данных онлайн-ритейлеров производит симуляцию этих же данных, используя для каждого товара еженедельное усреднение по ценам, что производит близкое соответствие данным Nielsen. Еженедельное усреднение приводит к тому, что одно изменение цен превращается в два последовательных изменения цен, что порождает более частые и меньшие по размеру изменения цен, что полностью меняет распределение ценовых изменений. Кроме того, это приводит к тому, что функция риска (важный динамический показатель поведения цен) искажается и становится полностью убывающей – с искусственным пиком изменений на первой неделе.

Автор показал, что средний период неизменности цен и размер изменений сокращается примерно на 50\%, и такой эффект наблюдается для всех стран. Этот результат объясняет полученные в литературе результаты обнаружения очень гибких цен. Автор обнаружил, что для США средний период неизменности цен составляет 1,53 месяца, против оценок на данных независимых агентств, варьирующихся от 0,6 до 1 месяца. Использование симуляции усреднения на онлайн-данных породило средний период изменения, равный 0,8 месяцам, что является средним значением между 0,6 и 1 месяцам. Чтобы сделать сравнение более точным, автором были приобретены данные независимого агентства для одного и того же ритейлера, локации и временного периода на онлайн-данных и данных этого агентства. Автор получил оценку в 0,8 месяцев, что соответствует усреднению по онлайн-данным для того же самого ритейлера, почтового индекса и временного периода.

Вместе с тем, среднему периоду изменений цен в литературе уделяется меньше внимания, чем распределению изменений цен. Автор показывает, что распределение изменений цен в онлайн-ритейле является в большей степени бимодальным, с очень малым количество изменений, близких к 0. Данные независимых агентств, напротив, демонстрируют одномодальное распределение изменений цен с очень высокой долей изменений, близких к 0. Автор отмечает, что такое распределение является превалирующим в литературе и послужило мотивацией к созданию моделей ценообразования, учитывающей малые изменения цен. Таким образом, еженедельное усреднение цен может объяснять различие в полученных выводах. Из-за усреднения цен, количество малых по модулю изменений становится значительно больше, что приводит к одномодальности распределения. Автор также подчеркивает, что малая доля околонулевых изменений и наличие двух мод соответствует предсказаниям гипотезе издержек меню, главной идеей которой является отсутствие малых изменений из-за их неоптимальности.

Наконец, усреднение во времени влияет на оценки функций риска. Функции риска показывают зависимость вероятности изменения цены от срока ее неизменности. Функция является важной с точки зрения определения того, какая из моделей ценообразования является более адекватной наблюдаемым данным. Модели с «издержками меню» предполагают, что функция риска является возрастающей, поскольку со временем все больше товаров оказываются за границами своего «бездействия» и выгоды от изменения цены вынуждают все большее число фирм принимать решение об изменении цен, таким образом вероятность изменения цен со временем растет. Модели, в которых ценообразование зависит от времени, напротив, предполагают либо отсутствие взаимосвязи между сроком изменения цен и вероятности изменения (как в модели Кальво), либо генерируют пики вероятности в определенные периоды «жизни» цены (как в модели Тейлора). Усреднение цен во времени приводит к тому, что функция риска становится строго убывающей, с высоким значением вероятности изменения цен в первую неделю. Напротив, функция риска, построенная на данных онлайн-ритейлеров имеет форму горба: сначала вероятность изменения цен растет со временем, затем постепенно падает. Увеличение вероятности изменения цен в начале функции также совпадает с выводами моделей издержек меню в том, что чем дольше цена остается неизменной, тем больше она отклоняется от оптимальной цены (в условиях устойчивых шоков, таких как накопленная инфляция), и тем выше становится вероятность изменения цен.

Авторы также остановились на эффектах, которые производит замещение пропущенных наблюдений по ценам на данных по расчету ИПЦ. Эти данные не подвержены усреднению во времени, как было описано выше, поскольку собираются один раз в месяц сотрудниками статистических ведомств. Однако, в случае отсутствия определенного товара в наличии, цена на него не регистрируется и заменяется определенным способом. Отсутствие наблюдений может быть связано с заменой товара на дргой товар (из-за его исчезновения из продажи) или его временной недоступностью (отсутствие на складе). 

Для замены пропущенных наблюдений многие статистические ведомства используются специальный метод – т.н. «относительное вменение по ячейкам» (cell-relative imputation). В рамках этого подхода цены на отсутствующий товар заменяются предыдущей ценой этого товара, умноженной на среднюю динамику внутри категории. Авторы указывают, что такой подход может механически приводить к увеличению частоты изменения цен, а также снижать размер этих изменений.

Чтобы проиллюстрировать эффект замещения пропущенных наблюдений средней динамикой по категории автор произвел симуляцию отсутствия наблюдений на данных онлайн-ритейлеров. Были отобраны цены на 15-ое число каждого месяца. Затем для каждого товара пропущенные цены были заменены предыдущей ценой, умноженной на геометрическое среднее изменений внутри той же категории товаров.

Результаты симуляции показали, что подход замещения пропущенных наблюдений драматически снижает дюрацию, или среднюю продолжительность неизменности цен. Ежемесячная дюрация сократилась с 4,7 месяцев до 3,35 месяцев, что близко к оценкам на данных, лежащих в основе расчета ИПЦ для США \cite{klenow2008state}. Таким образом, оценки частоты изменений цен на данных ИПЦ завышены из-за использования такого подхода замещения пропущенных наблюдений.

Эффект замещения пропущенных наблюдений также отражается и на размерах изменений цен. «Относительное вменение по ячейкам» приводит к тому, что средний размер изменений цен по модулю сокращается в США с 20,8\% до 16,2\%, а также увеличивается доля изменений цен ниже 1\% и 5\% по модулю, что также близко к оценкам \cite{klenow2008state}. В качестве дополнительной иллюстрации автор приводит график распределения изменений цен на ежемесячных онлайн-данных с симуляцией и без симуляции относительного вменения по ячейкам. Иллюстрация показывает, что, как и в случае с усреднением цен во времени, главным последствием замещения приводит к росту доли малых изменений цен. В случае использования такого замещения распределение цен становится одномодальным, тогда как без симуляции оно является бимодальным, что драматически искажает выводы о действительном поведении цен.

Также, как и в случае усреднения во времени, относительное вменение по ячейкам отражается и на функции риска порождением в ней ярко выраженного нисходящего наклона. Использование замещения пропущенных наблюдений приводит к возникновению двух последовательных изменений цен, что увеличивает вероятность изменения цены в первый месяц ее «жизни». 

Автор указывает, что выраженность эффектов смещения на настоящих данных ИПЦ может быть, однако, ниже, чем продемонстрировано им на симуляционных онлайн-данных. Одна из причин состоит в том, что доля пропущенных наблюдений в симуляции выше, чем заявленная в работе \cite{klenow2008state} (в последней эта доля равна 7\%). Во-вторых, не все пропущенные значения цен заменяются методикой «относительного вменения по ячейкам». Тем не менее, проведенная автором статьи симуляция указывает на то, что следует учитывать вышеуказанные особенности данных ИПЦ при оценке на их основе статистик жесткости цен.

\section{Эволюция исследований, посвященных использованию данных онлайн-ритейлеров для прогнозирования ценовых индексов}\label{sec:ch1/sec2}

Одним из направлений использования данных по ценам онлайн-ритейлеров стало прогнозирование офлайн-инфляции. Одной из первых работ, посвященных использованию онлайн-данных в таком ключе, стало исследование \cite{aparicio2020forecasting}. Авторы на данных онлайн-ритейлеров показали, что использование таких данных позволяет превзойти прогностические свойства различных традиционных офлайн-моделей, а также опросы профессиональных аналитиков.

Авторы использовали данные по ценам онлайн-ритейлеров компании PriceStats (коммерческое ответвление проекта The Billion Prices Project \cite{cavallo2016billion}) по следующим странам: Австралия, Канада, Франция, Германия, Греция, Ирландия, Италия, Нидерланды, Великобритания и США за период с июля 2008 г. по сентябрь 2016 г. Основная цель авторов состояла в предсказании значений месячного сезонно несглаженного индекса потребительских цен, рассчитанного для каждой из указанных стран. Авторы также указывают, что дневная частота данных онлайн-ритейлеров является преимуществом перед другими мерами инфляции, поскольку помогают обнаружить изменения в инфляционных трендах раньше конца месяца. Еще одно преимущество состоит в том, что индексы онлайн-ритейлеров публикуются сразу, тогда как как показатели ИПЦ обычно публикуются спустя 15 дней после сбора информации.

Модель, используемая авторами для прогнозирования ИПЦ, предлагается в следующем виде:

\begin{equation}
	\label{eq:equation2}
	E_{t-1}p_t = \hat{\alpha} + \sum_{i=1}^p \hat{\beta}_{t-i} p_{t-i} + \sum_{i=0}^p \hat{\theta}_{t-i} f_{t-i} + \sum_{i=0}^p \hat{\gamma}_{t-i} o_{t-i} + \sum_{i=0}^p \hat{\eta}_{t-i} of_{t-i},
\end{equation}
где \( p_t \) "--- индекс инфляции, \( f_t \) "--- индекс инфляции, построенный на данных по офлайн-ценам на бензин и дизельное топливо, \( o_t \) "--- онлайн-индекс инфляции, \( of_t \) "--- онлайн-индекс инфляции, построенный на данных по офлайн-ценам на бензин и дизельное топливо. Использование данных по дизельному топливу и бензину объясняется тем, что эти индексы по этим данным традиционно считаются хорошим предиктором для инфляции.

Оценка уравнения \cref{eq:equation2} сравнивается с 5 моделями-бенчмарками. Первая модель – это \cref{eq:equation2} без учета индексов, построенных на онлайн-данных. Сравнение с этой моделью позволяет оценить общее влияние добавления онлайн-индикаторов на прогнозирование инфляции. Второе сравнение проводится с прогнозами на 1 месяц вперед, опубликованным Bloomberg – прогноз, который является одним из самых известных прогнозов на рынке. Наконец, производится сравнение с тремя моделями, являющимися традиционными бенчмарками в литературе: AR(p), случайное блуждание (RW) и кривая Филиппса.

Авторы оценивают прогноз на один, два и три месяца вперед, и затем на основе этих прогнозов рассчитывают квартальный и годовой прогноз инфляции в США. Сравнение результатов модели \cref{eq:equation2} с результатами прогнозов вышеназванных бенчмарков показало, что качество прогнозов на основе модели с включением данных онлайн-ритейлеров значительно превосходит прогностические свойства моделей на традиционных офлайн-данных. Таким образом, в работе была впервые продемонстрирована полезность данных онлайн-ритейлеров для улучшения прогнозных свойств традиционных моделей прогнозирования инфляции.

     \begin{table}[h!]
     	\centering
     	\caption{Прогнозы инфляции}
     	\label{tab:inflation_forecasts}
     	\resizebox{\textwidth}{!}{%
     		\begin{tabular}{r|r|r|r|r|r|r}
     			\textbf{Онлайн} & \textbf{Офлайн} & \textbf{Средний Блумберг} & \textbf{Медиана Блумберг} & \textbf{AR(p)} & \textbf{Кривая Филлипса} & \textbf{RW} \\
     			\hline
     			0,429 & 0,806 & 1,677 & 1,785 & 1,198 & 1,968 & 1,138 \\
     		\end{tabular}
     	}
     \end{table}
     

Еще одним исследованием, посвященным использованию онлайн-данных по ценам для уточнения прогнозов по инфляции стала работа Национального банка Польши \cite{macias2023nowcasting}. Исследование является идейным продолжением работы \cite{aparicio2020forecasting}. Авторы фокусируются на предсказании продовольственной инфляции, поскольку доля продовольствия в потребительской корзине Польщи свыше 25\%, что является значительным вкладом в общий ИПЦ Польши.

Данные охватывают период с декабря 2009 г. по январь 2018 г. До 2016 года авторы собирали данные на еженедельной основе, а с 2016 г. – на ежедневной. Собранная база данных охватыват около 75 миллионов наблюдений цен на продовольственные товары, что покрывает примерно 488 тыс. товаров в 4-7 продуктовых онлайн-магазинах. Кроме того, для определения товаров по категориям была использована полуавтоматическая процедура разметки, в рамках которой сначала происходит разметка алгоритмами машинного обучения, затем производится ручной поиск ошибок разметки.

Для целей прогнозирования авторы агрегировали цены в месячную частоту. Авторы исследования использовали ту же модель, что и авторы \cite{aparicio2020forecasting}, однако без включения части, связанной с ценами на топливо. Производилась оценка 72 отдельных спецификаций (по одной на каждую продовольственную категорию). В качестве бенчмарков рассматривался авторегрессионный процесс (AR(p)), процесс скользящего среднего (RW), модель с сезонным случайным трендом, ARMA-модель и сезонная ARMA-модель. Результаты показали, что модель с онлайн-данными на коротком горизонте прогнозирования в среднем дает более точные прогнозы, чем в моделях без использования онлайн-данных. \cite{macias2023nowcasting}           % Глава 1
\chapter{Описание механизма сбора и обработки данных по ценам}\label{ch:ch2}
В России, как и во всем мире, в последние годы наблюдается тенденция к наращиванию объемов интернет-торговли. Данные Ассоциации компаний интернет-торговли показывают, что объем интернет-торговли в России за последние 9 лет вырос практически в 10 раз  --- с 713 млрд рублей в 2014 году до 6,3 трлн рублей в 2023 году. Развитие сегмента ускорилось в период пандемии, когда население оказалось в условиях коронавирусных ограничений, что привело к росту онлайн-покупок. Рост интернет-торговли привел также к росту данных по ценам на отдельные товары и услуги, доступные к покупке через интернет, что существенно расширило возможности исследователей для сбора данных и анализа поведения цен на микроуровне. В настоящей главе будет описана методика сбора данных, характеристики собранных данных, а также трудности, с которыми исследователь может столкнуться в процессе сбора данных и возможные пути решения этих трудностей с учетом накопленного опыта.
\section{Методика сбора данных по ценам онлайн-ритейлеров}\label{sec:ch2/sec1}

Механизм сбора данных работает следующим образом: исследователи совместно с техническими специалистами разрабатывают программные коды на Python, которые ежедневно запускаются на удаленном сервере. Эти программы автоматически собирают информацию о текущей цене товара, скидочной цене (если товар продается по сниженной цене), названии товара, ссылке на товар, а также в некоторых случаях о дополнительных характеристиках, таких как вес, цвет, размер упаковки и другие.

Одним из важных преимуществ онлайн-данных является их относительно низкая стоимость сбора. Написанная однажды программа и расходы на обслуживание сервера значительно дешевле, чем содержание штата сотрудников, которые должны регулярно посещать точки продаж и вносить цены в базу данных. Кроме того, вероятность совершения ошибки из-за человеческого фактора минимизируется, так как человек не участвует в процессе сбора данных. Программа автоматически собирает данные, однако требуется контроль за корректностью собранных данных. Это не вызывает существенных затрат в плане когнитивных и физических усилий. Важным преимуществом собранных данных является их высокая частотность, что позволяет отслеживать ценовые тенденции практически в режиме реального времени, особенно на площадках онлайн-ритейла.

Процесс сбора онлайн-данных о ценах устроен следующим образом. Ежедневно или с любой другой частотой программа анализирует и извлекает данные из разметки сайта, содержащей информацию о товаре и его стоимости. Если разметка меняется, возникает ошибка, требующая обновления кода. Следует отметить, что по опыту автора такие ошибки возникают редко. Затем информация обрабатывается и проверяется на ошибки, чтобы соответствовать формату базы данных, и заносится в нее.

Главным недостатком собранных онлайн-данных является малое покрытие категорий и точек продаж. Но последние исследования показывают, что это недостаток постепенно устраняется. Например, доля онлайн-продаж в общем обороте розничной торговли в России значительно увеличилась с 2\% в 2019 году до 9,2\% в 2021 году, согласно данным ассоциации интернет-торговли.

Выбор сайтов, с которых собирается информация, является важным аспектом. Если исследователь стремится определить тенденции в офлайн-сегменте по ценам в онлайн-ритейле, рекомендуется отдать предпочтение мультиканальным ритейлерам, которые предлагают продукцию как онлайн, так и в офлайн точках продаж. В таких случаях цены на товары обычно совпадают в 72\% случаев и изменения цен происходят с одинаковой частотой и амплитудой.

Для достижения репрезентативности важно учитывать размеры ритейлеров и выбирать те, которые занимают значительную долю в расходах потребителей. Также следует сосредоточиться на категориях товаров, входящих в традиционную корзину потребления домохозяйств.

Что касается выбора сайтов для сбора данных о товарах, рекомендуется предпочитать онлайн-ритейлеров, так как другие источники, такие как агрегаторы и третьи сайты, не всегда предоставляют актуальную информацию о товарах и их характеристиках. Нельзя гарантировать, что сделки проводятся по ценам, указанным на таких сайтах-агрегаторах.

\section{Проблемы сбора данных и возможные способы их решения}\label{sec:ch2/sec2}
           % Глава 2
\chapter{Основной анализ}\label{ch:ch3}

           % Глава 3
\chapter*{Заключение}                       % Заголовок
\addcontentsline{toc}{chapter}{Заключение}  % Добавляем его в оглавление

%% Согласно ГОСТ Р 7.0.11-2011:
%% 5.3.3 В заключении диссертации излагают итоги выполненного исследования, рекомендации, перспективы дальнейшей разработки темы.
%% 9.2.3 В заключении автореферата диссертации излагают итоги данного исследования, рекомендации и перспективы дальнейшей разработки темы.
%% Поэтому имеет смысл сделать эту часть общей и загрузить из одного файла в автореферат и в диссертацию:

%Основные результаты работы заключаются в следующем.
%\input{common/concl}
%И какая-нибудь заключающая фраза.
%
%Последний параграф может включать благодарности.  В заключение автор
%выражает благодарность и большую признательность научному руководителю
%Иванову~И.\,И. за поддержку, помощь, обсуждение результатов и~научное
%руководство. Также автор благодарит Сидорова~А.\,А. и~Петрова~Б.\,Б.
%за помощь в~работе с~образцами, Рабиновича~В.\,В. за предоставленные
%образцы и~обсуждение результатов, Занудятину~Г.\,Г. и авторов шаблона
%*Russian-Phd-LaTeX-Dissertation-Template* за~помощь в оформлении
%диссертации. Автор также благодарит много разных людей
%и~всех, кто сделал настоящую работу автора возможной.
      % Заключение
\include{Dissertation/acronyms}        % Список сокращений и условных обозначений
\chapter*{Словарь терминов}             % Заголовок
%\addcontentsline{toc}{chapter}{Словарь терминов}  % Добавляем его в оглавление
%
%\textbf{TeX} : Cистема компьютерной вёрстки, разработанная американским профессором информатики Дональдом Кнутом
%
%\textbf{панграмма} : Короткий текст, использующий все или почти все буквы алфавита
      % Словарь терминов
\include{Dissertation/references}      % Список литературы
\include{Dissertation/lists}           % Списки таблиц и изображений (иллюстративный материал)

\setcounter{totalchapter}{\value{chapter}} % Подсчёт количества глав

%%% Настройки для приложений
\appendix
% Оформление заголовков приложений ближе к ГОСТ:
\setlength{\midchapskip}{20pt}
\renewcommand*{\afterchapternum}{\par\nobreak\vskip \midchapskip}
\renewcommand\thechapter{\Asbuk{chapter}} % Чтобы приложения русскими буквами нумеровались

\chapter{Примеры вставки листингов программного кода}\label{app:A}

%Для крупных листингов есть два способа. Первый красивый, но в нём могут быть
%проблемы с поддержкой кириллицы (у вас может встречаться в~комментариях
%и~печатаемых сообщениях), он представлен на листинге~\cref{lst:hwbeauty}.
%\begin{ListingEnv}[!h]% настройки floating аналогичны окружению figure
%    \captiondelim{ } % разделитель идентификатора с номером от наименования
%    \caption{Программа ,,Hello, world`` на \protect\cpp}\label{lst:hwbeauty}
%    % окружение учитывает пробелы и табуляции и применяет их в сответсвии с настройками
%    \begin{lstlisting}[language={[ISO]C++}]
%	#include <iostream>
%	using namespace std;
%
%	int main() //кириллица в комментариях при xelatex и lualatex имеет проблемы с пробелами
%	{
%		cout << "Hello, world" << endl; //latin letters in commentaries
%		system("pause");
%		return 0;
%	}
%    \end{lstlisting}
%\end{ListingEnv}%
%Второй не~такой красивый, но без ограничений (см.~листинг~\cref{lst:hwplain}).
%\begin{ListingEnv}[!h]
%    \captiondelim{ } % разделитель идентификатора с номером от наименования
%    \caption{Программа ,,Hello, world`` без подсветки}\label{lst:hwplain}
%    \begin{Verb}
%
%        #include <iostream>
%        using namespace std;
%
%        int main() //кириллица в комментариях
%        {
%            cout << "Привет, мир" << endl;
%        }
%    \end{Verb}
%\end{ListingEnv}
%
%Можно использовать первый для вставки небольших фрагментов
%внутри текста, а второй для вставки полного
%кода в приложении, если таковое имеется.
%
%Если нужно вставить совсем короткий пример кода (одна или две строки),
%то~выделение  линейками и нумерация может смотреться чересчур громоздко.
%В таких случаях можно использовать окружения \texttt{lstlisting} или
%\texttt{Verb} без \texttt{ListingEnv}. Приведём такой пример
%с указанием языка программирования, отличного от~заданного по умолчанию:
%\begin{lstlisting}[language=Haskell]
%fibs = 0 : 1 : zipWith (+) fibs (tail fibs)
%\end{lstlisting}
%Такое решение "--- со вставкой нумерованных листингов покрупнее
%и~вставок без выделения для маленьких фрагментов "--- выбрано,
%например, в~книге Эндрю Таненбаума и Тодда Остина по архитектуре
%компьютера.
%
%Наконец, для оформления идентификаторов внутри строк
%(функция \lstinline{main} и~тому подобное) используется
%\texttt{lstinline} или, самое простое, моноширинный текст
%(\texttt{\textbackslash texttt}).
%
%Пример~\cref{lst:internal3}, иллюстрирующий подключение переопределённого
%языка. Может быть полезным, если подсветка кода работает криво. Без
%дополнительного окружения, с подписью и ссылкой, реализованной встроенным
%средством.
%\begingroup
%\captiondelim{ } % разделитель идентификатора с номером от наименования
%\begin{lstlisting}[language={Renhanced},caption={Пример листинга c подписью собственными средствами},label={lst:internal3}]
%## Caching the Inverse of a Matrix
%
%## Matrix inversion is usually a costly computation and there may be some
%## benefit to caching the inverse of a matrix rather than compute it repeatedly
%## This is a pair of functions that cache the inverse of a matrix.
%
%## makeCacheMatrix creates a special "matrix" object that can cache its inverse
%
%makeCacheMatrix <- function(x = matrix()) {#кириллица в комментариях при xelatex и lualatex имеет проблемы с пробелами
%    i <- NULL
%    set <- function(y) {
%        x <<- y
%        i <<- NULL
%    }
%    get <- function() x
%    setSolved <- function(solve) i <<- solve
%    getSolved <- function() i
%    list(set = set, get = get,
%    setSolved = setSolved,
%    getSolved = getSolved)
%
%}
%
%
%## cacheSolve computes the inverse of the special "matrix" returned by
%## makeCacheMatrix above. If the inverse has already been calculated (and the
%## matrix has not changed), then the cachesolve should retrieve the inverse from
%## the cache.
%
%cacheSolve <- function(x, ...) {
%    ## Return a matrix that is the inverse of 'x'
%    i <- x$getSolved()
%    if(!is.null(i)) {
%        message("getting cached data")
%        return(i)
%    }
%    data <- x$get()
%    i <- solve(data, ...)
%    x$setSolved(i)
%    i
%}
%\end{lstlisting} %$ %Комментарий для корректной подсветки синтаксиса
%%вне листинга
%\endgroup
%
%Листинг~\cref{lst:external1} подгружается из внешнего файла. Приходится
%загружать без окружения дополнительного. Иначе по страницам не переносится.
%\begingroup
%\captiondelim{ } % разделитель идентификатора с номером от наименования
%\lstinputlisting[lastline=78,language={R},caption={Листинг из внешнего файла},label={lst:external1}]{listings/run_analysis.R}
%\endgroup
%
%\chapter{Очень длинное название второго приложения, в~котором продемонстрирована работа с~длинными таблицами}\label{app:B}
%
%\section{Подраздел приложения}\label{app:B1}
%Вот размещается длинная таблица:
%\fontsize{10pt}{10pt}\selectfont
%\begin{longtable*}[c]{|l|c|l|l|} %longtable* появляется из пакета ltcaption и даёт ненумерованную таблицу
%    % \caption{Описание входных файлов модели}\label{Namelists}
%    %\\
%    \hline
%    %\multicolumn{4}{|c|}{\textbf{Файл puma\_namelist}}        \\ \hline
%    Параметр & Умолч. & Тип & Описание               \\ \hline
%    \endfirsthead   \hline
%    \multicolumn{4}{|c|}{\small\slshape (продолжение)}        \\ \hline
%    Параметр & Умолч. & Тип & Описание               \\ \hline
%    \endhead        \hline
%    % \multicolumn{4}{|c|}{\small\slshape (окончание)}        \\ \hline
%    % Параметр & Умолч. & Тип & Описание               \\ \hline
%    %                                             \endlasthead        \hline
%    \multicolumn{4}{|r|}{\small\slshape продолжение следует}  \\ \hline
%    \endfoot        \hline
%    \endlastfoot
%    \multicolumn{4}{|l|}{\&INP}        \\ \hline
%    kick & 1 & int & 0: инициализация без шума (\(p_s = const\)) \\
%    &   &     & 1: генерация белого шума                  \\
%    &   &     & 2: генерация белого шума симметрично относительно \\
%    & & & экватора    \\
%    mars & 0 & int & 1: инициализация модели для планеты Марс     \\
%    kick & 1 & int & 0: инициализация без шума (\(p_s = const\)) \\
%    &   &     & 1: генерация белого шума                  \\
%    &   &     & 2: генерация белого шума симметрично относительно \\
%    & & & экватора    \\
%    mars & 0 & int & 1: инициализация модели для планеты Марс     \\
%    kick & 1 & int & 0: инициализация без шума (\(p_s = const\)) \\
%    &   &     & 1: генерация белого шума                  \\
%    &   &     & 2: генерация белого шума симметрично относительно \\
%    & & & экватора    \\
%    mars & 0 & int & 1: инициализация модели для планеты Марс     \\
%    kick & 1 & int & 0: инициализация без шума (\(p_s = const\)) \\
%    &   &     & 1: генерация белого шума                  \\
%    &   &     & 2: генерация белого шума симметрично относительно \\
%    & & & экватора    \\
%    mars & 0 & int & 1: инициализация модели для планеты Марс     \\
%    kick & 1 & int & 0: инициализация без шума (\(p_s = const\)) \\
%    &   &     & 1: генерация белого шума                  \\
%    &   &     & 2: генерация белого шума симметрично относительно \\
%    & & & экватора    \\
%    mars & 0 & int & 1: инициализация модели для планеты Марс     \\
%    kick & 1 & int & 0: инициализация без шума (\(p_s = const\)) \\
%    &   &     & 1: генерация белого шума                  \\
%    &   &     & 2: генерация белого шума симметрично относительно \\
%    & & & экватора    \\
%    mars & 0 & int & 1: инициализация модели для планеты Марс     \\
%    kick & 1 & int & 0: инициализация без шума (\(p_s = const\)) \\
%    &   &     & 1: генерация белого шума                  \\
%    &   &     & 2: генерация белого шума симметрично относительно \\
%    & & & экватора    \\
%    mars & 0 & int & 1: инициализация модели для планеты Марс     \\
%    kick & 1 & int & 0: инициализация без шума (\(p_s = const\)) \\
%    &   &     & 1: генерация белого шума                  \\
%    &   &     & 2: генерация белого шума симметрично относительно \\
%    & & & экватора    \\
%    mars & 0 & int & 1: инициализация модели для планеты Марс     \\
%    kick & 1 & int & 0: инициализация без шума (\(p_s = const\)) \\
%    &   &     & 1: генерация белого шума                  \\
%    &   &     & 2: генерация белого шума симметрично относительно \\
%    & & & экватора    \\
%    mars & 0 & int & 1: инициализация модели для планеты Марс     \\
%    kick & 1 & int & 0: инициализация без шума (\(p_s = const\)) \\
%    &   &     & 1: генерация белого шума                  \\
%    &   &     & 2: генерация белого шума симметрично относительно \\
%    & & & экватора    \\
%    mars & 0 & int & 1: инициализация модели для планеты Марс     \\
%    kick & 1 & int & 0: инициализация без шума (\(p_s = const\)) \\
%    &   &     & 1: генерация белого шума                  \\
%    &   &     & 2: генерация белого шума симметрично относительно \\
%    & & & экватора    \\
%    mars & 0 & int & 1: инициализация модели для планеты Марс     \\
%    kick & 1 & int & 0: инициализация без шума (\(p_s = const\)) \\
%    &   &     & 1: генерация белого шума                  \\
%    &   &     & 2: генерация белого шума симметрично относительно \\
%    & & & экватора    \\
%    mars & 0 & int & 1: инициализация модели для планеты Марс     \\
%    kick & 1 & int & 0: инициализация без шума (\(p_s = const\)) \\
%    &   &     & 1: генерация белого шума                  \\
%    &   &     & 2: генерация белого шума симметрично относительно \\
%    & & & экватора    \\
%    mars & 0 & int & 1: инициализация модели для планеты Марс     \\
%    kick & 1 & int & 0: инициализация без шума (\(p_s = const\)) \\
%    &   &     & 1: генерация белого шума                  \\
%    &   &     & 2: генерация белого шума симметрично относительно \\
%    & & & экватора    \\
%    mars & 0 & int & 1: инициализация модели для планеты Марс     \\
%    kick & 1 & int & 0: инициализация без шума (\(p_s = const\)) \\
%    &   &     & 1: генерация белого шума                  \\
%    &   &     & 2: генерация белого шума симметрично относительно \\
%    & & & экватора    \\
%    mars & 0 & int & 1: инициализация модели для планеты Марс     \\
%    \hline
%    %& & & \(\:\) \\
%    \multicolumn{4}{|l|}{\&SURFPAR}        \\ \hline
%    kick & 1 & int & 0: инициализация без шума (\(p_s = const\)) \\
%    &   &     & 1: генерация белого шума                  \\
%    &   &     & 2: генерация белого шума симметрично относительно \\
%    & & & экватора    \\
%    mars & 0 & int & 1: инициализация модели для планеты Марс     \\
%    kick & 1 & int & 0: инициализация без шума (\(p_s = const\)) \\
%    &   &     & 1: генерация белого шума                  \\
%    &   &     & 2: генерация белого шума симметрично относительно \\
%    & & & экватора    \\
%    mars & 0 & int & 1: инициализация модели для планеты Марс     \\
%    kick & 1 & int & 0: инициализация без шума (\(p_s = const\)) \\
%    &   &     & 1: генерация белого шума                  \\
%    &   &     & 2: генерация белого шума симметрично относительно \\
%    & & & экватора    \\
%    mars & 0 & int & 1: инициализация модели для планеты Марс     \\
%    kick & 1 & int & 0: инициализация без шума (\(p_s = const\)) \\
%    &   &     & 1: генерация белого шума                  \\
%    &   &     & 2: генерация белого шума симметрично относительно \\
%    & & & экватора    \\
%    mars & 0 & int & 1: инициализация модели для планеты Марс     \\
%    kick & 1 & int & 0: инициализация без шума (\(p_s = const\)) \\
%    &   &     & 1: генерация белого шума                  \\
%    &   &     & 2: генерация белого шума симметрично относительно \\
%    & & & экватора    \\
%    mars & 0 & int & 1: инициализация модели для планеты Марс     \\
%    kick & 1 & int & 0: инициализация без шума (\(p_s = const\)) \\
%    &   &     & 1: генерация белого шума                  \\
%    &   &     & 2: генерация белого шума симметрично относительно \\
%    & & & экватора    \\
%    mars & 0 & int & 1: инициализация модели для планеты Марс     \\
%    kick & 1 & int & 0: инициализация без шума (\(p_s = const\)) \\
%    &   &     & 1: генерация белого шума                  \\
%    &   &     & 2: генерация белого шума симметрично относительно \\
%    & & & экватора    \\
%    mars & 0 & int & 1: инициализация модели для планеты Марс     \\
%    kick & 1 & int & 0: инициализация без шума (\(p_s = const\)) \\
%    &   &     & 1: генерация белого шума                  \\
%    &   &     & 2: генерация белого шума симметрично относительно \\
%    & & & экватора    \\
%    mars & 0 & int & 1: инициализация модели для планеты Марс     \\
%    kick & 1 & int & 0: инициализация без шума (\(p_s = const\)) \\
%    &   &     & 1: генерация белого шума                  \\
%    &   &     & 2: генерация белого шума симметрично относительно \\
%    & & & экватора    \\
%    mars & 0 & int & 1: инициализация модели для планеты Марс     \\
%    \hline
%\end{longtable*}
%
%\normalsize% возвращаем шрифт к нормальному
%\section{Ещё один подраздел приложения}\label{app:B2}
%
%Нужно больше подразделов приложения!
%Конвынёры витюпырата но нам, тебиквюэ мэнтётюм позтюлант ед про. Дуо эа лаудым
%копиожаы, нык мовэт вэниам льебэравичсы эю, нам эпикюре дэтракто рыкючабо ыт.
%
%Пример длинной таблицы с записью продолжения по ГОСТ 2.105:
%
%\begingroup
%\centering
%\small
%\captionsetup[table]{skip=7pt} % смещение положения подписи
%\begin{longtable}[c]{|l|c|l|l|}
%    \caption{Наименование таблицы средней длины}\label{tab:test5}% label всегда желательно идти после caption
%    \\[-0.45\onelineskip]
%    \hline
%    Параметр & Умолч. & Тип & Описание                                          \\ \hline
%    \endfirsthead%
%    \caption*{Продолжение таблицы~\thetable}                                    \\[-0.45\onelineskip]
%    \hline
%    Параметр & Умолч. & Тип & Описание                                          \\ \hline
%    \endhead
%    \hline
%    \endfoot
%    \hline
%    \endlastfoot
%    \multicolumn{4}{|l|}{\&INP}                                                 \\ \hline
%    kick     & 1      & int & 0: инициализация без шума (\(p_s = const\))       \\
%             &        &     & 1: генерация белого шума                          \\
%             &        &     & 2: генерация белого шума симметрично относительно \\
%             &        &     & экватора                                          \\
%    mars     & 0      & int & 1: инициализация модели для планеты Марс          \\
%    kick     & 1      & int & 0: инициализация без шума (\(p_s = const\))       \\
%             &        &     & 1: генерация белого шума                          \\
%             &        &     & 2: генерация белого шума симметрично относительно \\
%             &        &     & экватора                                          \\
%    mars     & 0      & int & 1: инициализация модели для планеты Марс          \\
%    kick     & 1      & int & 0: инициализация без шума (\(p_s = const\))       \\
%             &        &     & 1: генерация белого шума                          \\
%             &        &     & 2: генерация белого шума симметрично относительно \\
%             &        &     & экватора                                          \\
%    mars     & 0      & int & 1: инициализация модели для планеты Марс          \\
%    kick     & 1      & int & 0: инициализация без шума (\(p_s = const\))       \\
%             &        &     & 1: генерация белого шума                          \\
%             &        &     & 2: генерация белого шума симметрично относительно \\
%             &        &     & экватора                                          \\
%    mars     & 0      & int & 1: инициализация модели для планеты Марс          \\
%    kick     & 1      & int & 0: инициализация без шума (\(p_s = const\))       \\
%             &        &     & 1: генерация белого шума                          \\
%             &        &     & 2: генерация белого шума симметрично относительно \\
%             &        &     & экватора                                          \\
%    mars     & 0      & int & 1: инициализация модели для планеты Марс          \\
%    kick     & 1      & int & 0: инициализация без шума (\(p_s = const\))       \\
%             &        &     & 1: генерация белого шума                          \\
%             &        &     & 2: генерация белого шума симметрично относительно \\
%             &        &     & экватора                                          \\
%    mars     & 0      & int & 1: инициализация модели для планеты Марс          \\
%    kick     & 1      & int & 0: инициализация без шума (\(p_s = const\))       \\
%             &        &     & 1: генерация белого шума                          \\
%             &        &     & 2: генерация белого шума симметрично относительно \\
%             &        &     & экватора                                          \\
%    mars     & 0      & int & 1: инициализация модели для планеты Марс          \\
%    kick     & 1      & int & 0: инициализация без шума (\(p_s = const\))       \\
%             &        &     & 1: генерация белого шума                          \\
%             &        &     & 2: генерация белого шума симметрично относительно \\
%             &        &     & экватора                                          \\
%    mars     & 0      & int & 1: инициализация модели для планеты Марс          \\
%    kick     & 1      & int & 0: инициализация без шума (\(p_s = const\))       \\
%             &        &     & 1: генерация белого шума                          \\
%             &        &     & 2: генерация белого шума симметрично относительно \\
%             &        &     & экватора                                          \\
%    mars     & 0      & int & 1: инициализация модели для планеты Марс          \\
%    kick     & 1      & int & 0: инициализация без шума (\(p_s = const\))       \\
%             &        &     & 1: генерация белого шума                          \\
%             &        &     & 2: генерация белого шума симметрично относительно \\
%             &        &     & экватора                                          \\
%    mars     & 0      & int & 1: инициализация модели для планеты Марс          \\
%    kick     & 1      & int & 0: инициализация без шума (\(p_s = const\))       \\
%             &        &     & 1: генерация белого шума                          \\
%             &        &     & 2: генерация белого шума симметрично относительно \\
%             &        &     & экватора                                          \\
%    mars     & 0      & int & 1: инициализация модели для планеты Марс          \\
%    kick     & 1      & int & 0: инициализация без шума (\(p_s = const\))       \\
%             &        &     & 1: генерация белого шума                          \\
%             &        &     & 2: генерация белого шума симметрично относительно \\
%             &        &     & экватора                                          \\
%    mars     & 0      & int & 1: инициализация модели для планеты Марс          \\
%    kick     & 1      & int & 0: инициализация без шума (\(p_s = const\))       \\
%             &        &     & 1: генерация белого шума                          \\
%             &        &     & 2: генерация белого шума симметрично относительно \\
%             &        &     & экватора                                          \\
%    mars     & 0      & int & 1: инициализация модели для планеты Марс          \\
%    kick     & 1      & int & 0: инициализация без шума (\(p_s = const\))       \\
%             &        &     & 1: генерация белого шума                          \\
%             &        &     & 2: генерация белого шума симметрично относительно \\
%             &        &     & экватора                                          \\
%    mars     & 0      & int & 1: инициализация модели для планеты Марс          \\
%    kick     & 1      & int & 0: инициализация без шума (\(p_s = const\))       \\
%             &        &     & 1: генерация белого шума                          \\
%             &        &     & 2: генерация белого шума симметрично относительно \\
%             &        &     & экватора                                          \\
%    mars     & 0      & int & 1: инициализация модели для планеты Марс          \\
%    \hline
%    %& & & $\:$ \\
%    \multicolumn{4}{|l|}{\&SURFPAR}                                             \\ \hline
%    kick     & 1      & int & 0: инициализация без шума (\(p_s = const\))       \\
%             &        &     & 1: генерация белого шума                          \\
%             &        &     & 2: генерация белого шума симметрично относительно \\
%             &        &     & экватора                                          \\
%    mars     & 0      & int & 1: инициализация модели для планеты Марс          \\
%    kick     & 1      & int & 0: инициализация без шума (\(p_s = const\))       \\
%             &        &     & 1: генерация белого шума                          \\
%             &        &     & 2: генерация белого шума симметрично относительно \\
%             &        &     & экватора                                          \\
%    mars     & 0      & int & 1: инициализация модели для планеты Марс          \\
%    kick     & 1      & int & 0: инициализация без шума (\(p_s = const\))       \\
%             &        &     & 1: генерация белого шума                          \\
%             &        &     & 2: генерация белого шума симметрично относительно \\
%             &        &     & экватора                                          \\
%    mars     & 0      & int & 1: инициализация модели для планеты Марс          \\
%    kick     & 1      & int & 0: инициализация без шума (\(p_s = const\))       \\
%             &        &     & 1: генерация белого шума                          \\
%             &        &     & 2: генерация белого шума симметрично относительно \\
%             &        &     & экватора                                          \\
%    mars     & 0      & int & 1: инициализация модели для планеты Марс          \\
%    kick     & 1      & int & 0: инициализация без шума (\(p_s = const\))       \\
%             &        &     & 1: генерация белого шума                          \\
%             &        &     & 2: генерация белого шума симметрично относительно \\
%             &        &     & экватора                                          \\
%    mars     & 0      & int & 1: инициализация модели для планеты Марс          \\
%    kick     & 1      & int & 0: инициализация без шума (\(p_s = const\))       \\
%             &        &     & 1: генерация белого шума                          \\
%             &        &     & 2: генерация белого шума симметрично относительно \\
%             &        &     & экватора                                          \\
%    mars     & 0      & int & 1: инициализация модели для планеты Марс          \\
%    kick     & 1      & int & 0: инициализация без шума (\(p_s = const\))       \\
%             &        &     & 1: генерация белого шума                          \\
%             &        &     & 2: генерация белого шума симметрично относительно \\
%             &        &     & экватора                                          \\
%    mars     & 0      & int & 1: инициализация модели для планеты Марс          \\
%    kick     & 1      & int & 0: инициализация без шума (\(p_s = const\))       \\
%             &        &     & 1: генерация белого шума                          \\
%             &        &     & 2: генерация белого шума симметрично относительно \\
%             &        &     & экватора                                          \\
%    mars     & 0      & int & 1: инициализация модели для планеты Марс          \\
%    kick     & 1      & int & 0: инициализация без шума (\(p_s = const\))       \\
%             &        &     & 1: генерация белого шума                          \\
%             &        &     & 2: генерация белого шума симметрично относительно \\
%             &        &     & экватора                                          \\
%    mars     & 0      & int & 1: инициализация модели для планеты Марс          \\
%\end{longtable}
%\normalsize% возвращаем шрифт к нормальному
%\endgroup
%\section{Использование длинных таблиц с окружением \textit{longtabu}}\label{app:B2a}
%
%В таблице \cref{tab:test-functions} более книжный вариант
%длинной таблицы, используя окружение \verb!longtabu! и разнообразные
%\verb!toprule! \verb!midrule! \verb!bottomrule! из~пакета
%\verb!booktabs!. Чтобы визуально таблица смотрелась лучше, можно
%использовать следующие параметры: в самом начале задаётся расстояние
%между строчками с~помощью \verb!arraystretch!. Таблица задаётся на
%всю ширину, \verb!longtabu! позволяет делить ширину колонок
%пропорционально "--- тут три колонки в~пропорции 1.1:1:4 "--- для каждой
%колонки первый параметр в~описании \verb!X[]!. Кроме того, в~таблице
%убраны отступы слева и справа с~помощью \verb!@{}!
%в~преамбуле таблицы. К~первому и~второму столбцу применяется
%модификатор
%
%\verb!>{\setlength{\baselineskip}{0.7\baselineskip}}!,
%
%\noindent который уменьшает межстрочный интервал в для текста таблиц (иначе
%заголовок второго столбца значительно шире, а двухстрочное имя
%сливается с~окружающими). Для первой и второй колонки текст в ячейках
%выравниваются по~центру как по~вертикали, так и по горизонтали "---
%задаётся буквами \verb!m!~и~\verb!c!~в~описании столбца \verb!X[]!.
%
%Так как формулы большие "--- используется окружение \verb!alignedat!,
%чтобы отступ был одинаковый у всех формул "--- он сделан для всех, хотя
%для большей части можно было и не использовать.  Чтобы формулы
%занимали поменьше места в~каждом столбце формулы (где надо)
%используется \verb!\textstyle! "--- он~делает дроби меньше, у~знаков
%суммы и произведения "--- индексы сбоку. Иногда формула слишком большая,
%сливается со следующей, поэтому после неё ставится небольшой
%дополнительный отступ \verb!\vspace*{2ex}!. Для штрафных функций "---
%размер фигурных скобок задан вручную \verb!\Big\{!, т.\:к. не~умеет
%\verb!alignedat! работать с~\verb!\left! и~\verb!\right! через
%несколько строк/колонок.
%
%В примечании к таблице наоборот, окружение \verb!cases! даёт слишком
%большие промежутки между вариантами, чтобы их уменьшить, в конце
%каждой строчки окружения использовался отрицательный дополнительный
%отступ \verb!\\[-0.5em]!.
%
%\begingroup % Ограничиваем область видимости arraystretch
%\renewcommand{\arraystretch}{1.6}%% Увеличение расстояния между рядами, для улучшения восприятия.
%\begin{longtabu} to \textwidth
%    {%
%    @{}>{\setlength{\baselineskip}{0.7\baselineskip}}X[1.1mc]%
%    >{\setlength{\baselineskip}{0.7\baselineskip}}X[1.1mc]%
%    X[4]@{}%
%    }
%    \caption{Тестовые функции для оптимизации, \(D\) "---
%        размерность. Для всех функций значение в точке глобального
%        минимума равно нулю.\label{tab:test-functions}}\\% label всегда желательно идти после caption
%
%    \toprule     %%% верхняя линейка
%    Имя           &Стартовый диапазон параметров &Функция  \\
%    \midrule %%% тонкий разделитель. Отделяет названия столбцов. Обязателен по ГОСТ 2.105 пункт 4.4.5
%    \endfirsthead
%
%    \multicolumn{3}{c}{\small\slshape (продолжение)}        \\
%    \toprule     %%% верхняя линейка
%    Имя           &Стартовый диапазон параметров &Функция  \\
%    \midrule %%% тонкий разделитель. Отделяет названия столбцов. Обязателен по ГОСТ 2.105 пункт 4.4.5
%    \endhead
%
%    \multicolumn{3}{c}{\small\slshape (окончание)}        \\
%    \toprule     %%% верхняя линейка
%    Имя           &Стартовый диапазон параметров &Функция  \\
%    \midrule %%% тонкий разделитель. Отделяет названия столбцов. Обязателен по ГОСТ 2.105 пункт 4.4.5
%    \endlasthead
%
%    \bottomrule %%% нижняя линейка
%    \multicolumn{3}{r}{\small\slshape продолжение следует}  \\
%    \endfoot
%    \endlastfoot
%
%    сфера         &\(\left[-100,\,100\right]^D\)   &
%    \(\begin{aligned}
%        \textstyle f_1(x)=\sum_{i=1}^Dx_i^2
%    \end{aligned}\) \\
%    Schwefel 2.22 &\(\left[-10,\,10\right]^D\)     &
%    \(\begin{aligned}
%        \textstyle f_2(x)=\sum_{i=1}^D|x_i|+\prod_{i=1}^D|x_i|
%    \end{aligned}\) \\
%    Schwefel 1.2  &\(\left[-100,\,100\right]^D\)   &
%    \(\begin{aligned}
%        \textstyle f_3(x)=\sum_{i=1}^D\left(\sum_{j=1}^ix_j\right)^2
%    \end{aligned}\) \\
%    Schwefel 2.21 &\(\left[-100,\,100\right]^D\)   &
%    \(\begin{aligned}
%        \textstyle f_4(x)=\max_i\!\left\{\left|x_i\right|\right\}
%    \end{aligned}\) \\
%    Rosenbrock    &\(\left[-30,\,30\right]^D\)     &
%    \(\begin{aligned}
%        \textstyle f_5(x)=
%        \sum_{i=1}^{D-1}
%        \left[100\!\left(x_{i+1}-x_i^2\right)^2+(x_i-1)^2\right]
%    \end{aligned}\) \\
%    ступенчатая   &\(\left[-100,\,100\right]^D\)   &
%    \(\begin{aligned}
%        \textstyle f_6(x)=\sum_{i=1}^D\big\lfloor x_i+0.5\big\rfloor^2
%    \end{aligned}\) \\
%    зашумлённая квартическая &\(\left[-1.28,\,1.28\right]^D\) &
%    \(\begin{aligned}
%        \textstyle f_7(x)=\sum_{i=1}^Dix_i^4+rand[0,1)
%    \end{aligned}\)\vspace*{2ex}\\
%    Schwefel 2.26 &\(\left[-500,\,500\right]^D\)   &
%    \(\begin{aligned}
%        f_8(x)= & \textstyle\sum_{i=1}^D-x_i\,\sin\sqrt{|x_i|}\,+ \\
%                & \vphantom{\sum}+ D\cdot
%        418.98288727243369
%    \end{aligned}\)\\
%    Rastrigin     &\(\left[-5.12,\,5.12\right]^D\) &
%    \(\begin{aligned}
%        \textstyle f_9(x)=\sum_{i=1}^D\left[x_i^2-10\,\cos(2\pi x_i)+10\right]
%    \end{aligned}\)\vspace*{2ex}\\
%    Ackley        &\(\left[-32,\,32\right]^D\)     &
%    \(\begin{aligned}
%        f_{10}(x)= & \textstyle -20\, \exp\!\left(
%        -0.2\sqrt{\frac{1}{D}\sum_{i=1}^Dx_i^2} \right)- \\
%                   & \textstyle - \exp\left(
%            \frac{1}{D}\sum_{i=1}^D\cos(2\pi x_i)  \right)
%        + 20 + e
%    \end{aligned}\) \\
%    Griewank      &\(\left[-600,\,600\right]^D\) &
%    \(\begin{aligned}
%        f_{11}(x)= & \textstyle \frac{1}{4000}\sum_{i=1}^{D}x_i^2 -
%        \prod_{i=1}^D\cos\left(x_i/\sqrt{i}\right) +1
%    \end{aligned}\) \vspace*{3ex} \\
%    штрафная 1    &\(\left[-50,\,50\right]^D\)     &
%    \(\begin{aligned}
%        f_{12}(x)= & \textstyle \frac{\pi}{D}\Big\{ 10\,\sin^2(\pi y_1) +            \\
%                   & +\textstyle \sum_{i=1}^{D-1}(y_i-1)^2
%        \left[1+10\,\sin^2(\pi y_{i+1})\right] +                                     \\
%                   & +(y_D-1)^2 \Big\} +\textstyle\sum_{i=1}^D u(x_i,\,10,\,100,\,4)
%    \end{aligned}\) \vspace*{2ex} \\
%    штрафная 2    &\(\left[-50,\,50\right]^D\)     &
%    \(\begin{aligned}
%        f_{13}(x)= & \textstyle 0.1 \Big\{\sin^2(3\pi x_1) +            \\
%                   & +\textstyle \sum_{i=1}^{D-1}(x_i-1)^2
%        \left[1+\sin^2(3 \pi x_{i+1})\right] +                          \\
%                   & +(x_D-1)^2\left[1+\sin^2(2\pi x_D)\right] \Big\} + \\
%                   & +\textstyle\sum_{i=1}^D u(x_i,\,5,\,100,\,4)
%    \end{aligned}\)\\
%    сфера         &\(\left[-100,\,100\right]^D\)   &
%    \(\begin{aligned}
%        \textstyle f_1(x)=\sum_{i=1}^Dx_i^2
%    \end{aligned}\) \\
%    Schwefel 2.22 &\(\left[-10,\,10\right]^D\)     &
%    \(\begin{aligned}
%        \textstyle f_2(x)=\sum_{i=1}^D|x_i|+\prod_{i=1}^D|x_i|
%    \end{aligned}\) \\
%    Schwefel 1.2  &\(\left[-100,\,100\right]^D\)   &
%    \(\begin{aligned}
%        \textstyle f_3(x)=\sum_{i=1}^D\left(\sum_{j=1}^ix_j\right)^2
%    \end{aligned}\) \\
%    Schwefel 2.21 &\(\left[-100,\,100\right]^D\)   &
%    \(\begin{aligned}
%        \textstyle f_4(x)=\max_i\!\left\{\left|x_i\right|\right\}
%    \end{aligned}\) \\
%    Rosenbrock    &\(\left[-30,\,30\right]^D\)     &
%    \(\begin{aligned}
%        \textstyle f_5(x)=
%        \sum_{i=1}^{D-1}
%        \left[100\!\left(x_{i+1}-x_i^2\right)^2+(x_i-1)^2\right]
%    \end{aligned}\) \\
%    ступенчатая   &\(\left[-100,\,100\right]^D\)   &
%    \(\begin{aligned}
%        \textstyle f_6(x)=\sum_{i=1}^D\big\lfloor x_i+0.5\big\rfloor^2
%    \end{aligned}\) \\
%    зашумлённая квартическая &\(\left[-1.28,\,1.28\right]^D\) &
%    \(\begin{aligned}
%        \textstyle f_7(x)=\sum_{i=1}^Dix_i^4+rand[0,1)
%    \end{aligned}\)\vspace*{2ex}\\
%    Schwefel 2.26 &\(\left[-500,\,500\right]^D\)   &
%    \(\begin{aligned}
%        f_8(x)= & \textstyle\sum_{i=1}^D-x_i\,\sin\sqrt{|x_i|}\,+ \\
%                & \vphantom{\sum}+ D\cdot
%        418.98288727243369
%    \end{aligned}\)\\
%    Rastrigin     &\(\left[-5.12,\,5.12\right]^D\) &
%    \(\begin{aligned}
%        \textstyle f_9(x)=\sum_{i=1}^D\left[x_i^2-10\,\cos(2\pi x_i)+10\right]
%    \end{aligned}\)\vspace*{2ex}\\
%    Ackley        &\(\left[-32,\,32\right]^D\)     &
%    \(\begin{aligned}
%        f_{10}(x)= & \textstyle -20\, \exp\!\left(
%        -0.2\sqrt{\frac{1}{D}\sum_{i=1}^Dx_i^2} \right)- \\
%                   & \textstyle - \exp\left(
%            \frac{1}{D}\sum_{i=1}^D\cos(2\pi x_i)  \right)
%        + 20 + e
%    \end{aligned}\) \\
%    Griewank      &\(\left[-600,\,600\right]^D\) &
%    \(\begin{aligned}
%        f_{11}(x)= & \textstyle \frac{1}{4000}\sum_{i=1}^{D}x_i^2 -
%        \prod_{i=1}^D\cos\left(x_i/\sqrt{i}\right) +1
%    \end{aligned}\) \vspace*{3ex} \\
%    штрафная 1    &\(\left[-50,\,50\right]^D\)     &
%    \(\begin{aligned}
%        f_{12}(x)= & \textstyle \frac{\pi}{D}\Big\{ 10\,\sin^2(\pi y_1) +            \\
%                   & +\textstyle \sum_{i=1}^{D-1}(y_i-1)^2
%        \left[1+10\,\sin^2(\pi y_{i+1})\right] +                                     \\
%                   & +(y_D-1)^2 \Big\} +\textstyle\sum_{i=1}^D u(x_i,\,10,\,100,\,4)
%    \end{aligned}\) \vspace*{2ex} \\
%    штрафная 2    &\(\left[-50,\,50\right]^D\)     &
%    \(\begin{aligned}
%        f_{13}(x)= & \textstyle 0.1 \Big\{\sin^2(3\pi x_1) +            \\
%                   & +\textstyle \sum_{i=1}^{D-1}(x_i-1)^2
%        \left[1+\sin^2(3 \pi x_{i+1})\right] +                          \\
%                   & +(x_D-1)^2\left[1+\sin^2(2\pi x_D)\right] \Big\} + \\
%                   & +\textstyle\sum_{i=1}^D u(x_i,\,5,\,100,\,4)
%    \end{aligned}\)\\
%    \midrule%%% тонкий разделитель
%    \multicolumn{3}{@{}p{\textwidth}}{%
%    \vspace*{-3.5ex}% этим подтягиваем повыше
%    \hspace*{2.5em}% абзацный отступ - требование ГОСТ 2.105
%    Примечание "---  Для функций \(f_{12}\) и \(f_{13}\)
%    используется \(y_i = 1 + \frac{1}{4}(x_i+1)\)
%    и~$u(x_i,\,a,\,k,\,m)=
%        \begin{cases*}
%            k(x_i-a)^m,  & \( x_i >a \)            \\[-0.5em]
%            0,           & \( -a\leq x_i \leq a \) \\[-0.5em]
%            k(-x_i-a)^m, & \( x_i <-a \)
%        \end{cases*}
%    $
%    }\\
%    \bottomrule %%% нижняя линейка
%\end{longtabu}
%\endgroup
%
%\section{Форматирование внутри таблиц}\label{app:B3}
%
%В таблице \cref{tab:other-row} пример с чересстрочным
%форматированием. В~файле \verb+userstyles.tex+  задаётся счётчик
%\verb+\newcounter{rowcnt}+ который увеличивается на~1 после каждой
%строчки (как указано в преамбуле таблицы). Кроме того, задаётся
%условный макрос \verb+\altshape+ который выдаёт одно
%из~двух типов форматирования в~зависимости от чётности счётчика.
%
%В таблице \cref{tab:other-row} каждая чётная строчка "--- синяя,
%нечётная "--- с наклоном и~слегка поднята вверх. Визуально это приводит
%к тому, что среднее значение и~среднеквадратичное изменение
%группируются и хорошо выделяются взглядом в~таблице. Сохраняется
%возможность отдельные значения в таблице выделить цветом или
%шрифтом. К первому и второму столбцу форматирование не применяется
%по~сути таблицы, к шестому общее форматирование не~применяется для
%наглядности.
%
%Так как заголовок таблицы тоже считается за строчку, то перед ним (для
%первого, промежуточного и финального варианта) счётчик обнуляется,
%а~в~\verb+\altshape+ для нулевого значения счётчика форматирования
%не~применяется.
%
%\begingroup % Ограничиваем область видимости arraystretch
%\renewcommand\altshape{
%    \ifnumequal{\value{rowcnt}}{0}{
%        % Стиль для заголовка таблицы
%    }{
%        \ifnumodd{\value{rowcnt}}
%        {
%            \color{blue} % Cтиль для нечётных строк
%        }{
%            \vspace*{-0.7ex}\itshape} % Стиль для чётных строк
%    }
%}
%\newcolumntype{A}{>{\centering\begingroup\altshape}X[1mc]<{\endgroup}}
%\needspace{2\baselineskip}
%\renewcommand{\arraystretch}{0.9}%% Уменьшаем  расстояние между
%%% рядами, чтобы таблица не так много
%%% места занимала в дисере.
%\begin{longtabu} to \textwidth {@{}X[0.27ml]@{}X[0.7mc]@{}A@{}A@{}A@{}X[0.98mc]@{}>{\setlength{\baselineskip}{0.7\baselineskip}}A@{}A<{\stepcounter{rowcnt}}@{}}
%    % \begin{longtabu} to \textwidth {@{}X[0.2ml]X[1mc]X[1mc]X[1mc]X[1mc]X[1mc]>{\setlength{\baselineskip}{0.7\baselineskip}}X[1mc]X[1mc]@{}}
%    \caption{Длинная таблица с примером чересстрочного форматирования\label{tab:other-row}}\vspace*{1ex}\\% label всегда желательно идти после caption
%    % \vspace*{1ex}     \\
%
%    \toprule %%% верхняя линейка
%    \setcounter{rowcnt}{0} &Итера\-ции & JADE\texttt{++} & JADE & jDE & SaDE
%    & DE/rand /1/bin & PSO \\
%    \midrule %%% тонкий разделитель. Отделяет названия столбцов. Обязателен по ГОСТ 2.105 пункт 4.4.5
%    \endfirsthead
%
%    \multicolumn{8}{c}{\small\slshape (продолжение)} \\
%    \toprule %%% верхняя линейка
%    \setcounter{rowcnt}{0} &Итера\-ции & JADE\texttt{++} & JADE & jDE & SaDE
%    & DE/rand /1/bin & PSO \\
%    \midrule %%% тонкий разделитель. Отделяет названия столбцов. Обязателен по ГОСТ 2.105 пункт 4.4.5
%    \endhead
%
%    \multicolumn{8}{c}{\small\slshape (окончание)} \\
%    \toprule %%% верхняя линейка
%    \setcounter{rowcnt}{0} &Итера\-ции & JADE\texttt{++} & JADE & jDE & SaDE
%    & DE/rand /1/bin & PSO \\
%    \midrule %%% тонкий разделитель. Отделяет названия столбцов. Обязателен по ГОСТ 2.105 пункт 4.4.5
%    \endlasthead
%
%    \bottomrule %%% нижняя линейка
%    \multicolumn{8}{r}{\small\slshape продолжение следует}     \\
%    \endfoot
%    \endlastfoot
%
%    f1  & 1500 & \textbf{1.8E-60}   & 1.3E-54   & 2.5E-28   & 4.5E-20   & 9.8E-14   & 9.6E-42   \\\nopagebreak
%    &      & (8.4E-60) & (9.2E-54) & {\color{red}(3.5E-28)} & (6.9E-20) & (8.4E-14) & (2.7E-41) \\
%    f2  & 2000 & 1.8E-25   & 3.9E-22   & 1.5E-23   & 1.9E-14   & 1.6E-09   & 9.3E-21   \\\nopagebreak
%    &      & (8.8E-25) & (2.7E-21) & (1.0E-23) & (1.1E-14) & (1.1E-09) & (6.3E-20) \\
%    f3  & 5000 & 5.7E-61   & 6.0E-87   & 5.2E-14   & {\color{green}9.0E-37}   & 6.6E-11   & 2.5E-19   \\\nopagebreak
%    &      & (2.7E-60) & (1.9E-86) & (1.1E-13) & (5.4E-36) & (8.8E-11) & (3.9E-19) \\
%    f4  & 5000 & 8.2E-24   & 4.3E-66   & 1.4E-15   & 7.4E-11   & 4.2E-01   & 4.4E-14   \\\nopagebreak
%    &      & (4.0E-23) & (1.2E-65) & (1.0E-15) & (1.8E-10) & (1.1E+00) & (9.3E-14) \\
%    f5  & 3000 & 8.0E-02   & 3.2E-01   & 1.3E+01   & 2.1E+01   & 2.1E+00   & 2.5E+01   \\\nopagebreak
%    &      & (5.6E-01) & (1.1E+00) & (1.4E+01) & (7.8E+00) & (1.5E+00) & (3.2E+01) \\
%    f6  & 100  & 2.9E+00   & 5.6E+00   & 1.0E+03   & 9.3E+02   & 4.7E+03   & 4.5E+01   \\\nopagebreak
%    &      & (1.2E+00) & (1.6E+00) & (2.2E+02) & (1.8E+02) & (1.1E+03) & (2.4E+01) \\
%    f7  & 3000 & 6.4E-04   & 6.8E-04   & 3.3E-03   & 4.8E-03   & 4.7E-03   & 2.5E-03   \\\nopagebreak
%    &      & (2.5E-04) & (2.5E-04) & (8.5E-04) & (1.2E-03) & (1.2E-03) & (1.4E-03) \\
%    f8  & 1000 & 3.3E-05   & 7.1E+00   & 7.9E-11   & 4.7E+00   & 5.9E+03   & 2.4E+03   \\\nopagebreak
%    &      & (2.3E-05) & (2.8E+01) & (1.3E-10) & (3.3E+01) & (1.1E+03) & (6.7E+02) \\
%    f9  & 1000 & 1.0E-04   & 1.4E-04   & 1.5E-04   & 1.2E-03   & 1.8E+02   & 5.2E+01   \\\nopagebreak
%    &      & (6.0E-05) & (6.5E-05) & (2.0E-04) & (6.5E-04) & (1.3E+01) & (1.6E+01) \\
%    f10 & 500  & 8.2E-10   & 3.0E-09   & 3.5E-04   & 2.7E-03   & 1.1E-01   & 4.6E-01   \\\nopagebreak
%    &      & (6.9E-10) & (2.2E-09) & (1.0E-04) & (5.1E-04) & (3.9E-02) & (6.6E-01) \\
%    f11 & 500  & 9.9E-08   & 2.0E-04   & 1.9E-05   & 7.8E-04  & 2.0E-01   & 1.3E-02   \\\nopagebreak
%    &      & (6.0E-07) & (1.4E-03) & (5.8E-05) & (1.2E-03)  & (1.1E-01) & (1.7E-02) \\
%    f12 & 500  & 4.6E-17   & 3.8E-16   & 1.6E-07   & 1.9E-05   & 1.2E-02   & 1.9E-01   \\\nopagebreak
%    &      & (1.9E-16) & (8.3E-16) & (1.5E-07) & (9.2E-06) & (1.0E-02) & (3.9E-01) \\
%    f13 & 500  & 2.0E-16   & 1.2E-15   & 1.5E-06   & 6.1E-05   & 7.5E-02   & 2.9E-03   \\\nopagebreak
%    &      & (6.5E-16) & (2.8E-15) & (9.8E-07) & (2.0E-05) & (3.8E-02) & (4.8E-03) \\
%    f1  & 1500 & \textbf{1.8E-60}   & 1.3E-54   & 2.5E-28   & 4.5E-20   & 9.8E-14   & 9.6E-42   \\\nopagebreak
%    &      & (8.4E-60) & (9.2E-54) & {\color{red}(3.5E-28)} & (6.9E-20) & (8.4E-14) & (2.7E-41) \\
%    f2  & 2000 & 1.8E-25   & 3.9E-22   & 1.5E-23   & 1.9E-14   & 1.6E-09   & 9.3E-21   \\\nopagebreak
%    &      & (8.8E-25) & (2.7E-21) & (1.0E-23) & (1.1E-14) & (1.1E-09) & (6.3E-20) \\
%    f3  & 5000 & 5.7E-61   & 6.0E-87   & 5.2E-14   & 9.0E-37   & 6.6E-11   & 2.5E-19   \\\nopagebreak
%    &      & (2.7E-60) & (1.9E-86) & (1.1E-13) & (5.4E-36) & (8.8E-11) & (3.9E-19) \\
%    f4  & 5000 & 8.2E-24   & 4.3E-66   & 1.4E-15   & 7.4E-11   & 4.2E-01   & 4.4E-14   \\\nopagebreak
%    &      & (4.0E-23) & (1.2E-65) & (1.0E-15) & (1.8E-10) & (1.1E+00) & (9.3E-14) \\
%    f5  & 3000 & 8.0E-02   & 3.2E-01   & 1.3E+01   & 2.1E+01   & 2.1E+00   & 2.5E+01   \\\nopagebreak
%    &      & (5.6E-01) & (1.1E+00) & (1.4E+01) & (7.8E+00) & (1.5E+00) & (3.2E+01) \\
%    f6  & 100  & 2.9E+00   & 5.6E+00   & 1.0E+03   & 9.3E+02   & 4.7E+03   & 4.5E+01   \\\nopagebreak
%    &      & (1.2E+00) & (1.6E+00) & (2.2E+02) & (1.8E+02) & (1.1E+03) & (2.4E+01) \\
%    f7  & 3000 & 6.4E-04   & 6.8E-04   & 3.3E-03   & 4.8E-03   & 4.7E-03   & 2.5E-03   \\\nopagebreak
%    &      & (2.5E-04) & (2.5E-04) & (8.5E-04) & (1.2E-03) & (1.2E-03) & (1.4E-03) \\
%    f8  & 1000 & 3.3E-05   & 7.1E+00   & 7.9E-11   & 4.7E+00   & 5.9E+03   & 2.4E+03   \\\nopagebreak
%    &      & (2.3E-05) & (2.8E+01) & (1.3E-10) & (3.3E+01) & (1.1E+03) & (6.7E+02) \\
%    f9  & 1000 & 1.0E-04   & 1.4E-04   & 1.5E-04   & 1.2E-03   & 1.8E+02   & 5.2E+01   \\\nopagebreak
%    &      & (6.0E-05) & (6.5E-05) & (2.0E-04) & (6.5E-04) & (1.3E+01) & (1.6E+01) \\
%    f10 & 500  & 8.2E-10   & 3.0E-09   & 3.5E-04   & 2.7E-03   & 1.1E-01   & 4.6E-01   \\\nopagebreak
%    &      & (6.9E-10) & (2.2E-09) & (1.0E-04) & (5.1E-04) & (3.9E-02) & (6.6E-01) \\
%    f11 & 500  & 9.9E-08   & 2.0E-04   & 1.9E-05   & 7.8E-04  & 2.0E-01   & 1.3E-02   \\\nopagebreak
%    &      & (6.0E-07) & (1.4E-03) & (5.8E-05) & (1.2E-03)  & (1.1E-01) & (1.7E-02) \\
%    f12 & 500  & 4.6E-17   & 3.8E-16   & 1.6E-07   & 1.9E-05   & 1.2E-02   & 1.9E-01   \\\nopagebreak
%    &      & (1.9E-16) & (8.3E-16) & (1.5E-07) & (9.2E-06) & (1.0E-02) & (3.9E-01) \\
%    f13 & 500  & 2.0E-16   & 1.2E-15   & 1.5E-06   & 6.1E-05   & 7.5E-02   & 2.9E-03   \\\nopagebreak
%    &      & (6.5E-16) & (2.8E-15) & (9.8E-07) & (2.0E-05) & (3.8E-02) & (4.8E-03) \\
%    \bottomrule %%% нижняя линейка
%\end{longtabu} \endgroup
%
%\section{Стандартные префиксы ссылок}\label{app:B4}
%
%Общепринятым является следующий формат ссылок: \texttt{<prefix>:<label>}.
%Например, \verb+\label{fig:knuth}+; \verb+\ref{tab:test1}+; \verb+label={lst:external1}+.
%В~таблице \cref{tab:tab_pref} приведены стандартные префиксы для различных
%типов ссылок.
%
%\begin{table}[htbp]
%    \captionsetup{justification=centering}
%    \centering{
%        \caption{\label{tab:tab_pref}Стандартные префиксы ссылок}
%        \begin{tabular}{ll}
%            \toprule
%            \textbf{Префикс} & \textbf{Описание} \\
%            \midrule
%            ch:              & Глава             \\
%            sec:             & Секция            \\
%            subsec:          & Подсекция         \\
%            fig:             & Рисунок           \\
%            tab:             & Таблица           \\
%            eq:              & Уравнение         \\
%            lst:             & Листинг программы \\
%            itm:             & Элемент списка    \\
%            alg:             & Алгоритм          \\
%            app:             & Секция приложения \\
%            \bottomrule
%        \end{tabular}
%    }
%\end{table}
%
%
%Для упорядочивания ссылок можно использовать разделительные символы.
%Например, \verb+\label{fig:scheemes/my_scheeme}+ или \\ \verb+\label{lst:dts/linked_list}+.
%
%\section{Очередной подраздел приложения}\label{app:B5}
%
%Нужно больше подразделов приложения!
%
%\section{И ещё один подраздел приложения}\label{app:B6}
%
%Нужно больше подразделов приложения!
%
%\clearpage
%\refstepcounter{chapter}
%\addcontentsline{toc}{appendix}{\protect\chapternumberline{\thechapter}Чертёж детали}
%
%\includepdf[pages=-]{Dissertation/images/drawing.pdf}
        % Приложения

\setcounter{totalappendix}{\value{chapter}} % Подсчёт количества приложений

\end{document}
